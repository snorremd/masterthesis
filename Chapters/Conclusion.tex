%!TEX root = ../Thesis.tex
\chapter{Summary and Conclusion} % Main chapter title
\label{Conclusion} % Change X to a consecutive number; for referencing this chapter elsewhere, use \ref{ChapterX}
\lhead{Chapter \ref{Conclusion}. \emph{Conclusion}} % Change X to a consecutive number; this is for the header on each page - perhaps a shortened title
The goal of the thesis was to identify an approach for finding optimized parameter sets for the \CTC algorithm. This goal was motivated by the need of researchers to better fit the parameters of the \CTC algorithm to the corpora of their research. This need was evident in the case of the ``Klimauken'' research project. The thesis propose and implement a genetic algorithm for parameter optimization. This algorithm was tested experimentally by forming a hypothesis: The genetic algorithm gives values that for the ``klimauken'' corpus give better performance than the average performance of random values.

\section{Results}
The experiment was performed based on the traditional information retrieval research methodology. Precision, recall, and the harmonius F-Measure was chosen as independent variables and measured for each tested parameter set with its correspondiThe approximate average performance of the algorithm was found by calculating the performance of 100 random parameter sets. The genetic parameter produced many optimised parameter set, but the chosen set was chosen for best possible F-Measure.

\begin{table}[H]
\setstretch{1}
\begin{center}
\begin{tabular}{|l|ccc|}
\hline
Test & Precision 0 & Recall 0 & F-Measure 0\\ 
\hline
Random 						&   0.422& 	  0.202& 	0.188\\ 
Genetic 					&   0.691&    0.783&    0.735\\ 
Point-wise 					&   0.522&    0.833&    0.642\\ 
\hline
\end{tabular}
\end{center}
\caption{Table summary of results from relevant tests.}
\label{tab:summarytableresultsconclusion}
\end{table}

The results are more than conclusive. The genetic parameter set is better than the average (random) performance with regards to all the measures. The precision and recall are respectively 26.9 and 58.1 percentile points higher than the same scores for the average performance of the algorithm. The genetic parameter set produces a harmonious F-Measure that is 54.7 percentile points higher. That is an approximate 390\% increase in score. The null-hypothesis was rejected, and the alternative hypothesis confirmed.

The results ufortunately do not possess external validity. This is because the tests were run on a single news corpus. We can there-fore not claim that the genetic algorithm will be able to optimise parameters for all news corpora. The results are not checked for statistical significance, but it is safe to assume that the genetic algorithm will be able to produce fairly consistent results with only small variance in terms of running time and exact parameter values.

\section{Future research}
To solidify the findings research should be conducted in order to confirm external validity for the proposed optimization algorithm. Such a study should test the algorithm on all standard news corpora such as the Reuters RCV1 corpus. Further more, it is likely that there is some overlap between the characteristics of news corpora. A possible area of research would include an investigation into whether the optimises parameter set will produce consistently better results on all news corpora.

One factor in clustering that was not accounted for in this research is time efficiency. The \CTC algorithm has an obvious bottle-neck in the base cluster merging step. New research using the genetic optimisation algorithm should factor in time when running the algorithm. Finding a good balance between high F-Measure and high time efficiency, and encoding this into the fitness measure of the genetic algorithm, can be a complex matter in itself.

While this thesis suggest that the proposed optimization algorithm works for news corpora, it is more uncertain whether it would also work for other kinds of corpora. A possible research effort could look into the effectiveness of the optimization algorithm for other types of corpora, for example the Blogs08 corpus.

\section{Conclusion}
Final remarks.