%!TEX root = ../Thesis.tex

% Chapter Template
\chapter{Introduction} % Main chapter title

\label{Introduction}
%use \ref{Introduction}

\lhead{Chapter \ref{Introduction}. \emph{Introduction}} % Change X to a consecutive number; this is for the header on each page - perhaps a shortened title

%----------------------------------------------------------------------------------------
%	SECTION 1: Background
%----------------------------------------------------------------------------------------

% \section{Background} Should this be used as the main introduction before the motivation? ala Aleksander Larsen
Information Retrieval is a field in information, informatics, and computer science that revolves around retrieving and classifying information in order to make information more accessible. Theories and practices developed in this field drive many of the information retrieval systems we use in our every day lives such as Google Search, Duckduckgo, TinEye (an image search engine), file system search engines, and many more. One major area of information retrieval is classification and grouping (or clustering) of text documents or other information artefacts. Clustering will be the subject of this master thesis.

This master thesis started as a master thesis proposal by \supervisor. The proposal suggests that work be put into optimising the effectiveness (quality of results) of the \STC algorithm by means of adjusting its parameters. The \STC algorithm is presented by \textcite{Oren1998} in the paper \citetitle{Oren1998}. The algorithm is a clustering algorithm which extracts phrases from text documents and finds clusters by inserting these phrases into a suffix tree. \cite{Moe2014compact} have worked with the \STC algorithm in relation to a research project at the \deptname. The project is described in the paper \citetitle{Elgesem2009}, \cite{Elgesem2009}.

In this work they experimented with the \STC algorithm, and its parameters, and implemented a variation called \CTC, which does not limit itself to suffixes. The original \STC algorithm proposed by \textcite{Oren1998} is demonstrated in use on search engine results and showed good results on these kinds of data. In their work with a sample corpus (henceforth called ``Klimauken'' corpus), a corpus comprising articles from several big Norwegian newspapers, \cite{Moe2014compact} found that the \STC algorithm yielded poorer results. This can be explained as search engine results have been pre-filtered by the search engine and thus show some similarities from the get go. The news documents are quite varied in content and as such the default implementation and parameters of the \STC algorithm are not sufficient to get good results. This provides the background for this master thesis. In the thesis an approach to finding a more optimised parameter sets for the \CTC algorithm will be explored.

%----------------------------------------------------------------------------------------
%	SECTION 2: Motivation
%----------------------------------------------------------------------------------------

\section{Motivation}

The \STC algorithm has one benefit over many of the traditional clustering algorithms. It is phrase based which means it takes into account the position of words when comparing the similarity of documents. Traditional clustering algorithms often use the bag of words model where each document is considered a set of words and any positional properties are disregarded. The drawback is however that the \STC algorithm has shown poor performance on an unfiltered text corpus. It is of interest to see if the algorithm's performance can be improved. There is evidence to suggest that an adaptation of the algorithm and its parameters might improve results. This thesis will propose an automated optimisation process as a way in which to find better optimised parameter sets for different corpora. The thesis will test this process by comparing the effectiveness (quality of results) of the parameter set produced by the optimisation process to the algorithm's average effectiveness.

The optimisation process, if successful, could benefit researchers who use the algorithm as part of their work to identify text clusters.

% \subsection{Other motivations}
% I have previously taken two courses about information retrieval and web intelligence. This area of research is very interesting and this master thesis provided an opportunity to learn more about it. I also wanted to use previous knowledge in my master thesis, so I proposed to use the \GA to identify good parameter sets as genetic algorithms are well suited to the task of exploring large feature spaces. I also got the opportunity to learn a new programming language, Python, as this was already used by \supervisor.

%----------------------------------------------------------------------------------------
%	SECTION 3: Research Questions
%----------------------------------------------------------------------------------------

\section{Research question}

The main goal of this master thesis is to identify an approach for finding optimized parameter sets for the \CTC algorithm. The number of possible parameter sets are very large, so there is no efficient way to computationally prove that the most optimal parameter set has been found. Instead the thesis will identify an optimised parameter set that performs better than randomly chosen parameter sets, and suggests an approach to finding such optimised parameter sets. An experiment will be performed to verify the results.

The thesis has three subsidiary goals:
\begin{itemize}
	\item Develop a method that optimise parameters for the Compact Trie Clustering algorithm.
	\item Apply the method to determine recommended parameter values for news documents.
	\item Demonstrate the feasibility of the method by applying the method to the ``Klimauken'' corpus.
\end{itemize}

This thesis is for researchers and users of the \STC algorithm. With the optimisation approach discussed in this thesis they will be able to find better parameter sets for their corpora. Demonstrating the feasibility of the optimisation method is therefore important. An optimisation algorithm that requires too much processing power to work would be quite impractical to use. The thesis is also very much aimed at the research group working with the project discussed in \citetitle{Elgesem2009} and other information retrieval researchers.

This research is performed as part of a master thesis. As such certain temporal and economic constraints are imposed upon the scope and detail of the research. The algorithm is tested on one corpus, the ``Klimauken'' corpus. The tests were initially run on various personal machines which made time efficiency comparisons between parameter sets difficult. The time element of optimisation is however an interesting one, and could be investigated further in future research.