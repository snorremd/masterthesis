% Chapter Template

\chapter{Introduction} % Main chapter title

\label{Introduction}
%use \ref{Introduction}

\lhead{Chapter \ref{Introduction}. \emph{Introduction}} % Change X to a consecutive number; this is for the header on each page - perhaps a shortened title

%----------------------------------------------------------------------------------------
%	SECTION 1: Background
%----------------------------------------------------------------------------------------

% \section{Background} Should this be used as the main introduction before the motivation? ala Aleksander Larsen
Introduce the background for the master thesis work. What is information retrieval, why is it used. Shortly talk about clustering and classification and how it is used to help make order in large document collections. What kind of problems might clustering solve?

Introduce the Klimauken project, and the compact trie clustering algorithm.


%----------------------------------------------------------------------------------------
%	SECTION 2: Motivation
%----------------------------------------------------------------------------------------

\section{Motivation}
Academic motivations:

The Compact Trie Clustering algorithm has some benefits (being computationally fast, supporting phrases), but performs poorly on data sets which have not been pre-filtered by say a search engine. How can this master thesis work contribute to the performance/effectivity of the algorithm. Why is this a interesting area to research.

Other motivations:

Information retrieval an interesting field. Genetic algorithms an interesting means to the optimization of large parameter sets. Computing fitness with genetic algorithm a challenging and exiting task. The opportunity to learn a new programming language (Python).

%----------------------------------------------------------------------------------------
%	SECTION 3: Research Questions
%----------------------------------------------------------------------------------------

\section{Research question}
The main research question of this thesis is:

\emph{"What are the optimal parameter values for Compact Trie Clustering with regard to the news corpus?"}

Explanation of research question here...

A list of subsidiary goals here:

\begin{itemize}
	\item Develop a  method/software for determining optimal parameters for the Compact Trie Clustering algorithm.
	\item Apply the method/software to determine recommended parameter values for news documents.
\end{itemize}

Why is this an interesting research question, and how does it relate to the motivation.

Target audience:
Researchers and users of the suffix tree/compact trie clustering algorithm in general, and the research group working with Klimauken and other information retrieval research in particular.

What are the limitations imposed upon the research. I.e. scope of research, tests performed, number of corpora used etc.