%%%%%%%%%%%%%%%%%%%%%%%%%%%%%%%%%%%%%%%%%
% Thesis
% LaTeX Template
% Version 1.2 (29/7/12)
%
% This template has been downloaded from:
% http://www.latextemplates.com
%
% Original authors:
% Steven Gunn
% http://users.ecs.soton.ac.uk/srg/softwaretools/document/templates/
% and
% Sunil Patel
% http://www.sunilpatel.co.uk/thesis-template/
%
% License:
% CC BY-NC-SA 3.0 (http://creativecommons.org/licenses/by-nc-sa/3.0/)
%
% Note:
% Make sure to edit document variables in the Thesis.cls file
%
%%%%%%%%%%%%%%%%%%%%%%%%%%%%%%%%%%%%%%%%%

%----------------------------------------------------------------------------------------
%	PACKAGES AND OTHER DOCUMENT CONFIGURATIONS
%----------------------------------------------------------------------------------------
%!TEX encoding = UTF-8 Unicode
\documentclass[11pt, a4paper, oneside]{Thesis} % Paper size, default font size and one-sided paper
\usepackage[T1]{fontenc}
\usepackage[utf8]{inputenc}
\DeclareUnicodeCharacter{00A0}{~}
\graphicspath{{./Figures/}} % Specifies the directory where pictures are stored

\usepackage[square, comma, sort&compress]{natbib} % Use the natbib reference package - read up on this to edit the reference style; if you want text (e.g. Smith et al., 2012) for the in-text references (instead of numbers), remove 'numbers'

% Package for writing graph
\usepackage{pgfplots}

% To add highlighting
\usepackage{soul}

% Packages for use with nomenclature
\usepackage[intoc]{nomencl}
\makenomenclature
%\renewcommand{\nomname}{List of Abbreviations}
\usepackage{mfirstuc} % Added for function which makes the first character of a word upper case
\newcommand*{\nom}[2]{#1\nomenclature{\makefirstuc{#1}}{#2}}

\newcommand*{\fig}[2]{
	\begin{figure}[ht]
		\begin{center}
			\includegraphics[width=0.7\textwidth]{#1.png}
			\caption{#2}
			\label{#1}
		\end{center}
	\end{figure}
}

\usepackage{listings}
\renewcommand*{\lstlistlistingname}{List of Listings}
\usepackage{color}
\definecolor{lightgray}{rgb}{.9,.9,.9}
\definecolor{darkgray}{rgb}{.4,.4,.4}
\definecolor{purple}{rgb}{0.65, 0.12, 0.82}

\lstdefinelanguage{JavaScript}{
  keywords={typeof, new, true, false, catch, function, return, null, catch, switch, var, if, in, while, do, else, case, break},
  keywordstyle=\color{blue}\bfseries,
  ndkeywords={class, export, boolean, throw, implements, import, this},
  ndkeywordstyle=\color{darkgray}\bfseries,
  identifierstyle=\color{black},
  sensitive=false,
  comment=[l]{//},
  morecomment=[s]{/*}{*/},
  commentstyle=\color{purple}\ttfamily,
  stringstyle=\color{red}\ttfamily,
  morestring=[b]',
  morestring=[b]"
}

\lstset{
   language=JavaScript,
   backgroundcolor=\color{lightgray},
   extendedchars=true,
   basicstyle=\footnotesize\ttfamily,
   showstringspaces=false,
   showspaces=false,
   numbers=left,
   numberstyle=\footnotesize,
   numbersep=9pt,
   tabsize=2,
   breaklines=true,
   showtabs=false,
   captionpos=b
}

%\hypersetup{hidelinks=true} % Go to http://en.wikibooks.org/wiki/LaTeX/Hyperlinks for information about configuration
\title{\ttitle} % Defines the thesis title - don't touch this
\def\theartefact{MaDaME}
\def\Theartefact{MaDaME}
\def\THEARTEFACT{MADAME}
\def\website{\url{http://madame.elseth.me}}
\begin{document}

\frontmatter % Use roman page numbering style (i, ii, iii, iv...) for the pre-content pages

\setstretch{1.5} % Line spacing of 1.3

% Define the page headers using the FancyHdr package and set up for one-sided printing
\fancyhead{} % Clears all page headers and footers
\rhead{\thepage} % Sets the right side header to show the page number
\lhead{} % Clears the left side page header

\pagestyle{fancy} % Finally, use the "fancy" page style to implement the FancyHdr headers

\newcommand{\HRule}{\rule{\linewidth}{0.5mm}} % New command to make the lines in the title page



% PDF meta-data
\hypersetup{pdftitle={\ttitle}}
\hypersetup{pdfsubject=\subjectname}
\hypersetup{pdfauthor=\authornames}
\hypersetup{pdfkeywords=\keywordnames}

%----------------------------------------------------------------------------------------
%	TITLE PAGE
%----------------------------------------------------------------------------------------

\begin{titlepage}
\begin{center}
\includegraphics[width=8cm]{Figures/uib-emblem-svart} \\[0.5cm]
% \includegraphics{uib-emblem-svart}\vfill % University/department logo - uncomment to place it
\textsc{\LARGE \univname}\\[1.5cm] % University name
\textsc{\Large Master Thesis}\\[0.5cm] % Thesis type

\HRule \\[0.4cm] % Horizontal line
{\huge \bfseries \ttitle}\\[0.4cm] % Thesis title
\HRule \\[1.5cm] % Horizontal line

\begin{minipage}{0.4\textwidth}
\begin{flushleft} \large
\emph{Author:}\\
\href{http://snorre.io}{\authornames} % Author name - remove the \href bracket to remove the link
\end{flushleft}
\end{minipage}
\begin{minipage}{0.4\textwidth}
\begin{flushright} \large
\emph{Supervisor:} \\
{\supname} % Supervisor name - remove the \href bracket to remove the link
\end{flushright}
\end{minipage}\\[2cm]

% \large \textit{A thesis submitted in fulfilment of the requirements\\ for the degree of \degreename}\\[0.3cm] % University requirement text
\textit{in the}\\[0.1cm]
\groupname\\\deptname\\[0.5cm] % Research group name and department name

{\large \today}\\[1cm] % Date
\end{center}

\end{titlepage}

%----------------------------------------------------------------------------------------
%	QUOTATION PAGE
%----------------------------------------------------------------------------------------

\pagestyle{empty} % No headers or footers for the following pages

\null\vfill % Add some space to move the quote down the page a bit

\textit{``It depends upon what the meaning of the word 'is' is"}

\begin{flushright}
-- Bill Clinton
\end{flushright}

\vfill\vfill\vfill\vfill\vfill\vfill\null % Add some space at the bottom to position the quote just right

\clearpage % Start a new page

%----------------------------------------------------------------------------------------
%	ABSTRACT PAGE
%----------------------------------------------------------------------------------------

\addtotoc{Abstract} % Add the "Abstract" page entry to the Contents

\abstract{\addtocontents{toc}{\vspace{1em}} % Add a gap in the Contents, for aesthetics
Abstract comes here!

}

\clearpage % Start a new page

%----------------------------------------------------------------------------------------
%	ACKNOWLEDGEMENTS
%----------------------------------------------------------------------------------------

\setstretch{1.5} % Reset the line-spacing to 1.3 for body text (if it has changed)

\acknowledgements{\addtocontents{toc}{\vspace{1em}} % Add a gap in the Contents, for aesthetics
Acknowledgements!
}
\clearpage % Start a new page

%----------------------------------------------------------------------------------------
%	LIST OF CONTENTS/FIGURES/TABLES/NOMENCLATURES PAGES
%----------------------------------------------------------------------------------------

\pagestyle{fancy} % The page style headers have been "empty" all this time, now use the "fancy" headers as defined before to bring them back

\lhead{\emph{Contents}} % Set the left side page header to "Contents"
\tableofcontents % Write out the Table of Contents

\lhead{\emph{List of Figures}} % Set the left side page header to "List of Figures"
\listoffigures % Write out the List of Figures

\lhead{\emph{List of Tables}} % Set the left side page header to "List of Tables"
\listoftables % Write out the List of Tables

\lhead{\emph{List of Listings}} % Set the left side page header to "List of Listings"
\addcontentsline{toc}{chapter}{List of Listings}
\lstlistoflistings

\lhead{\emph{Nomenclature}} % Set the left side page header to "List of Listings"
\printnomenclature[5em] % Print the page header and list of nomenclature

%----------------------------------------------------------------------------------------
%	DEDICATION
%----------------------------------------------------------------------------------------

% \setstretch{1.5} % Return the line spacing back to 1.3

% \pagestyle{empty} % Page style needs to be empty for this page

% \dedicatory{For/Dedicated to/To my\ldots} % Dedication text

% \addtocontents{toc}{\vspace{2em}} % Add a gap in the Contents, for aesthetics

%----------------------------------------------------------------------------------------
%	THESIS CONTENT - CHAPTERS
%----------------------------------------------------------------------------------------

\mainmatter % Begin numeric (1,2,3...) page numbering

\pagestyle{fancy} % Return the page headers back to the "fancy" style

% Include the chapters of the thesis as separate files from the Chapters folder
% Uncomment the lines as you write the chapters

%!TEX root = ../Thesis.tex

% Chapter Template
\chapter{Introduction} % Main chapter title

\label{Introduction}
%use \ref{Introduction}

\lhead{Chapter \ref{Introduction}. \emph{Introduction}} % Change X to a consecutive number; this is for the header on each page - perhaps a shortened title

%----------------------------------------------------------------------------------------
%	SECTION 1: Background
%----------------------------------------------------------------------------------------

% \section{Background} Should this be used as the main introduction before the motivation? ala Aleksander Larsen
Information Retrieval is a field in information, informatics, and computer science that revolves around retrieving and classifying information in order to make information more accessible. Theories and practices developed in this field drive many of the information retrieval systems we use in our every day lives such as Google Search, Duckduckgo, TinEye (an image search engine), file system search engines, and many more. One major area of information retrieval is classification and grouping (or clustering) of text documents or other information artefacts. Clustering will be the subject of this master thesis.

This master thesis started as a master thesis proposal by \supervisor. The proposal suggests that work be put into optimising the effectiveness (quality of results) of the \STC algorithm by means of adjusting its parameters. The \STC algorithm is presented by \textcite{Oren1998} in the paper \citetitle{Oren1998}. The algorithm is a clustering algorithm which extracts phrases from text documents and finds clusters by inserting these phrases into a suffix tree. \cite{Moe2014compact} have worked with the \STC algorithm in relation to a research project at the \deptname. The project is described in the paper \citetitle{Elgesem2009}, \cite{Elgesem2009}.

In this work they experimented with the \STC algorithm, and its parameters, and implemented a variation called \CTC, which does not limit itself to suffixes. The original \STC algorithm proposed by \textcite{Oren1998} is demonstrated in use on search engine results and showed good results on these kinds of data. In their work with a sample corpus (henceforth called ``Klimauken'' corpus), a corpus comprising articles from several big Norwegian newspapers, \cite{Moe2014compact} found that the \STC algorithm yielded poorer results. This can be explained as search engine results have been pre-filtered by the search engine and thus show some similarities from the get go. The news documents are quite varied in content and as such the default implementation and parameters of the \STC algorithm are not sufficient to get good results. This provides the background for this master thesis. In the thesis an approach to finding a more optimised parameter sets for the \CTC algorithm will be explored.

%----------------------------------------------------------------------------------------
%	SECTION 2: Motivation
%----------------------------------------------------------------------------------------

\section{Motivation}

The \STC algorithm has one benefit over many of the traditional clustering algorithms. It is phrase based which means it takes into account the position of words when comparing the similarity of documents. Traditional clustering algorithms often use the bag of words model where each document is considered a set of words and any positional properties are disregarded. The drawback is however that the \STC algorithm has shown poor performance on an unfiltered text corpus. It is of interest to see if the algorithm's performance can be improved. There is evidence to suggest that an adaptation of the algorithm and its parameters might improve results. This thesis will propose an automated optimisation process as a way in which to find better optimised parameter sets for different corpora. The thesis will test this process by comparing the effectiveness (quality of results) of the parameter set produced by the optimisation process to the algorithm's average effectiveness.

The optimisation process, if successful, could benefit researchers who use the algorithm as part of their work to identify text clusters.

% \subsection{Other motivations}
% I have previously taken two courses about information retrieval and web intelligence. This area of research is very interesting and this master thesis provided an opportunity to learn more about it. I also wanted to use previous knowledge in my master thesis, so I proposed to use the \GA to identify good parameter sets as genetic algorithms are well suited to the task of exploring large feature spaces. I also got the opportunity to learn a new programming language, Python, as this was already used by \supervisor.

%----------------------------------------------------------------------------------------
%	SECTION 3: Research Questions
%----------------------------------------------------------------------------------------

\section{Research question}

The main goal of this master thesis is to identify an approach for finding optimized parameter sets for the \CTC algorithm. The number of possible parameter sets are very large, so there is no efficient way to computationally prove that the most optimal parameter set has been found. Instead the thesis will identify an optimised parameter set that performs better than randomly chosen parameter sets, and suggests an approach to finding such optimised parameter sets. An experiment will be performed to verify the results.

The thesis has three subsidiary goals:
\begin{itemize}
	\item Develop a method that optimise parameters for the Compact Trie Clustering algorithm.
	\item Apply the method to determine recommended parameter values for news documents.
	\item Demonstrate the feasibility of the method by applying the method to the ``Klimauken'' corpus.
\end{itemize}

This thesis is for researchers and users of the \STC algorithm. With the optimisation approach discussed in this thesis they will be able to find better parameter sets for their corpora. Demonstrating the feasibility of the optimisation method is therefore important. An optimisation algorithm that requires too much processing power to work would be quite impractical to use. The thesis is also very much aimed at the research group working with the project discussed in \citetitle{Elgesem2009} and other information retrieval researchers.

This research is performed as part of a master thesis. As such certain temporal and economic constraints are imposed upon the scope and detail of the research. The algorithm is tested on one corpus, the ``Klimauken'' corpus. The tests were initially run on various personal machines which made time efficiency comparisons between parameter sets difficult. The time element of optimisation is however an interesting one, and could be investigated further in future research.
% Chapter Template

\chapter{Theory} % Main chapter title
\label{Theory} % Change X to a consecutive number; for referencing this chapter elsewhere, use \ref{ChapterX}
\lhead{Chapter \ref{Theory}. \emph{Theory}} % Change X to a consecutive number; this is for the header on each page - perhaps a shortened title

Introduction to chapter. I will rewrite and in general use the theory section from the project proposal.

\section{Clustering and information retrieval}
\label{Clustering}
A general introduction to text document clustering should be introduced here.

\subsection{Suffix Trees and Suffix Tree Clustering}
What is a suffix tree? How is one built?

\subsection{Compact Trie Clustering}
What is the compact trie clustering algorithm...

\subsection{Performance measures}
What kind of performance measures are used for clustering...

\subsection{Available corpora}
Which corpora are used in clustering and/or classification research? Which ones are suited to clustering? Which ones are used in this master thesis research? Explain scope...


\section{Genetic Algorithms}
\label{GeneticAlgorithm}
General overview of a genetic algorithm here.


\section{Related Work}
\label{RelatedWork}
Introduce related research work in this chapter.
%!TEX root = ../Thesis.tex
% Chapter Template

\chapter{Methodology} % Main chapter title

\label{Methodology} % Change X to a consecutive number; for referencing this chapter elsewhere, use \ref{ChapterX}

\lhead{Chapter \ref{Methodology}. \emph{Methodology}} % Change X to a consecutive number; this is for the header on each page - perhaps a shortened title

\color{red}TODO: The methodology and testing chapters are somewhat intertwined. The incrementally optimized parameter set is listed under algorithmic design in the methodology chapter, but how this parameter set is derived is not explained fully before the testing and evaluation chapter. Perhaps the incremental test section in the test chapter should be moved to the pre-experimental planning section in this methodology chapter.

The methodology chapter should be written in past tense?

\color{black}

Research methodologies and standards used to test retrieval and classification algorithms form the foundation of this thesis' research methodology. While experiments in the information retrieval field do not necessarily directly involve human subjects, there are still standards and methodologies in place to ensure that experimental results are valid. This chapter will describe the scientific approach used in the scientific field of information retrieval, and how this approach was applied to this thesis work.

\section{Experimental Evaluation}
\label{ExperimentalEvaluation}
The configuration and parameter sets for the \CTC algorithm was evaluated experimentally rather than analytically for reasons explained by \cite[][32]{Sebastiani2002}:
\begin{quote}The reason is that, in order to evaluate a system analytically (e.g., proving that the system is correct and complete), we would need a formal specification of the problem that the system is trying to solve (e.g., with respect to what correctness and completeness are defined), and the central notion of TC [Text Classification] (namely, that of membership of a document in a category) is, due to its subjective character, inherently nonformalizable. The experimental evaluation of a classifier usually measures its effectiveness (rather than its efficiency), that is, its ability to take the right classification decisions.
\end{quote}

Experimental evaluation have long been used and discussed in the field of information retrieval. One of the first experimental paradigms in information retrieval research, one that is still in use today, is the test collection evaluation paradigm introduced by The Cranfield research projects during the 60s \cite{Cleverdon1966}. The experimental methodology formed during these experiments are nicely summarized by \cite[][564]{Voorhees2005} who writes that:

\begin{quote}
At the core of this experimental methodology was the idea that live users could be removed from the evaluation loop, thus simplifying the evaluation and allowing researchers to run in vitro–style experiments in a laboratory with just their retrieval engine, a set of queries, a test collection, and a set of judgments (i.e., a list of relevant documents).
\end{quote}.

\cite[p. 33]{Cleverdon1966} give three requirements for using measurements of performance in experimental tests of information retrieval systems:
\begin{enumerate}
\item A document collection of known size to be used in the test;
\item A set of questions, together with decisions as to exactly which documents are relevant to each question;
\item A set of results of searches made in the test; these usually give the numbers of documents retrieved in the searches, divided into the relevant and non-relevant documents.
\end{enumerate}
These questions should be asked with regards to information retrieval experiments done on text search via queries, but can inspire similar questions for experiments done on text classification and clustering algorithms. Instead of forming questions and determining which documents are relevant to those question, one can form a set of categories and then decide which documents fall into which categories. Then clustering results can be divided into clusters, each cluster correct or incorrect.

\section{Corpora}
\label{Corpora}

When performing experimental research on information retrieval systems it is customary to use standard document collections or corpora as they are also known. There are several corpora available for text classification and clustering research. \cite{Baeza-Yates2011a} describe some of the corpora available for text classification research among them: Reuters-21578, RCV: Reuters Corpus Volumes, the OHSUMED reference collection and 20 NewsGroups. Some of them are briefly explained below.

Reuters, an international news agency have made several corpora that are available trough different sources. One Reuters corpus that have been much used in the text classification community is the \textit{Reuters 21578} corpus \cite{Lewis2004a}. The documents in this collection was collected from documents appearing on the Reuters newswire in 1987. The corpus was assembled and categorized by personnel from Reuters and Carnegie Group in 1987. This corpus thus resembles that used in the ``Recycling of news in the news papers 1999 - 2000'' research project.

Reuters have since made a new corpus that is likely to replace the Reuters 21578 corpus. This new corpus is divided into three volumes RCV1, RCV2 and TRC2. The first two volumes contain news stories from 96 - 97, and the last volume contains news stories from 08 to 09. RCV1 and TRC2 contain English news stories only, while RCV2 is multilingual \cite{NationalInstituteofStandardsandTechnology2004}. An article on use of the RCV1 corpus provide more details about how the data set can be used in evaluation text categorization systems. ``\textit{Reuters CorpusVolume I (RCV1) is an archive of over 800,000 manually categorized newswire stories recently made available by Reuters, Ltd. for research purposes. Use of this data for research on text categorization requires a detailed understanding of the real world constraints under which the data was produced.}'' \cite{Lewis2004}. 

This thesis focus mainly on news corpora, but the research group investigating the ``Klimauken'' corpus has also looked into the blogosphere. With this in mind it could be interesting to use a standard blog collection for evaluation of the algorithm in future research. Two blog collections are used in the blog track in the TREC conference, the Blogs06 and Blogs08 collections.\begin{quote}
The TREC Blogs06 collection is a big sample of the blogosphere, and contains spam as well as possibly non-blogs, e.g. RSS feeds from news broadcasters. It was crawled over an eleven week period from 6th December 2005 until the 21st February 2006. The collection is 148GB in size [\dots] The collection was used in TREC 2006, 2007 and 2008 \cite{Macdonald2011}.
\end{quote} 

These corpora form a solid foundation for experimental evaluations and make it possible to replicate and compare results between research projects and researchers. But for this to be possible, it is necessary to use some formal evaluation measures that are employed by a majority of the research in the area of study. This will be the focus of the next section (Section~\ref{EvaluationMeasures}).

\section{Evaluation Measures}
\label{EvaluationMeasures}
There are some considerations to take when choosing evaluation measures. When performing experimental research it is important to use the same evaluation measures as those used in related research works to make the results comparable. In much of information retrieval research, text classification included, there are agreed upon measures that can be used while performing research. Such measures do not exist to the same extent for clustering algorithms because the measure of relevance may not be similar in different clustering research. For the \STC algorithm there seems to be a community of practice with regards to evaluation metrics. The measure of relevance in this research is quite similar to that used in much classification work where each document is assigned a class.

\cite{Chim2007} detail how they perform an experimental evaluation of their clustering result in some detail. \citeauthor{Chim2007} use the evaluation measures on the results from a hierarchical agglomerative clustering algorithm as well as the \STC algorithm. These formulas have also been used by \cite{Rafi2011} when they compare the standard suffix tree clustering algorithm of \citeauthor{Oren1998} with the algorithm formulated by \citeauthor{Chim2008}. Their papers describe four standard measurements for clustering quality: precision, recall, F-Measure and overall F-Measure.

If you have the sets

\begin{displaymath}
C = \{C_{1}, C_{2}, \dots, C_{k}\}
\end{displaymath}
\begin{displaymath}
C^* = \{C_1^*, C_2^*, \dots, C_l^*\}
\end{displaymath}
\begin{displaymath}
D = \{D_{1}, D_{2}, \dots, D_{k}\}
\end{displaymath}

where \(C\) is the clusters produced by the algorithms on document set \(D\), and \(C^*\) is the ``correct'' classes of document set \(D\), then the recall, precision and F-measure of cluster \(j\) with respect to class \(i\) can be calculated as:

\begin{displaymath}
recall(i,j) = \frac{\vert C_{j} \cap C_i^* \vert}{\vert C_i^* \vert}
\end{displaymath}
\begin{displaymath}
precision(i,j) = \frac{\vert C_{j} \cap C_i^* \vert}{\vert C_{j} \vert}
\end{displaymath}
\begin{displaymath}
F-Measure(i,j) = \frac{2 \cdot precision(i,j) \cdot recall(i,j)}{precision(i,j) + recall(i,j)}
\end{displaymath}

\cite{Chim2007} do not provide any information as to whether the classes of \(C^*\) must be disjoint (i.e. whether one document can be assigned to multiple classes). The authors use this performance measure on both the \STC and Hierarchical Agglomerative Clustering algorithms. The \STC algorithm allows clusters to be overlapping, i.e. a document can occur in more than one document. The Hierarchical Agglomerative algorithm builds non-overlapping clusters. Based on this information one can assume that the authors have used the above performance measurements on both the overlapping and non-overlapping cluster sets.

As was explained in the literature section recall aims to capture the fraction of positive results to the total number of correct results. In this context recall expresses the fraction of a category's documents the cluster contains. Precision shows the fraction of documents in a cluster that is correctly clustered given a category to the amount of documents in a cluster. Because precision and recall is not good measures by themselves (recall could be a 100\%, but often precision would then be very low) an F-Measure is often used in evaluation of text classifiers \cite{Baeza-Yates2011a}. The F-Measure is the harmonic mean between recall and precision and is high when both precision and recall is high \cite{Baeza-Yates2011b}. 

The precision, recall, and F-Measure measurements defined above are applied to one cluster and class at a time. In other words the F-Measure of a cluster j with regards to a class i might not be any good, but its F-Measure with regards to another class i' might be very good. \cite{Chim2007} define an overall F-Measure function that captures the F-Measures for all the correct classes defined for the document set. For this function only those clusters j which maximize the F-Measure score for class i are considered in the overall F-Measure score. The overall F-Measure is calculated using the function:

Given the formulas the overall F-Measure of the clusters \(C\) can be calculated using the formula:
\begin{displaymath}
F := \sum_{i=1}^{l} \frac{\vert C_i^* \vert}{\vert D \vert} \cdot \max_{j=1,\dots,k} \{F-Measure(i,j)\}
\end{displaymath}

\subsection{News articles project}
The ``Klimauken'' corpus is tagged with five topics (tags or classes) per document rather than a single topic. See the tags attribute in Listing~\ref{lst:snippetfile} for an example of a document in snippet format. In context of this corpus, a ground truth cluster is defined as all those documents that have all five topics in common. But only looking at perfect matches might exclude those clusters that closely match a ground truth cluster. It thus makes sense to adjust the relevance definition and talk about degrees of correctness. To this end a measure, discrepancy, is introduced. The discrepancy of a cluster \(C\) with regards to a ground truth cluster \(G\) is defined as \(5 - d\) where \(d\) is the number of tags in common in the set \(C_{tags} \cup G_{tags}\). If all documents in the cluster and ground truth cluster have all five topics in common, there is no discrepancy. Further we impose a limit on discrepancy, namely that \(C\) matches ground truth with discrepancy \(d\) iff there is a \(G\) such that:

\begin{displaymath}
\{ G \vert G \subseteq C \textit{ and } G = max\{d(G_{1}), \dots, d(G_{n})\}\}
\end{displaymath}

The discrepancy of a cluster in relation to the ground truth is thus its discrepancy to the best matching ground truth cluster. An example of a cluster is given in Table~\ref{tab:clusterexample}.

\subsubsection{Precision}
With a definition of discrepancy in place we understand precision to be the number of ground truth clusters of zero discrepancy divided by the number of retrieved clusters. The division of retrieved clusters into levels of discrepancy allows us to relax the original measure and for example allow ground truth clusters of discrepancy zero and one to be used when calculating precision. For a clustering result with many clusters of discrepancy 1, including these clusters when calculating the precision can have a big impact.

\subsubsection{Recall}

The recall of a clustering result \(C\), given ground truth clusters \(G\) and a discrepancy \(d\), can be calculated: \(Recall = \frac{\vert G_{d} \vert}{\vert C \vert}\). As with precision looking only at ground truth clusters with a discrepancy of zero gives us the recall value given a traditional relevance measurement. Again one can use the accumulated recall values of discrepancy levels zero and one to include mostly correct clusters as ground truth clusters.

\subsubsection{Tag Accuracy}
A third measure called tag accuracy can also be used to meaure the clustering result. This measure leaves behind the notion of ground truth all together in favor of measuring the coherency between tags in the documents in the resulting clusters. \textit{``A cluster has a tag accuracy of 5 - d if and only if d is the number of words common to all tags in it.''}, \cite{Moe2013}. Given two documents with the tags 
\begin{inparaenum}[\itshape 1\upshape)]
	\item Innenriks-ulykker-akeulykke-8åring-nordkapp, and
	\item Innenriks-kriminalitet-trafikk-førerkort-kristiansand,
\end{inparaenum}
we get a tag accuracy of 4.


\begin{table}[htdp]
\footnotesize
%\rowcolors{1}{gray!10}{white}
\begin{center}
\begin{tabular}{|c|p{10cm} |}
\hline
Label &  smører Odd-Bjørn \\
Topics overlap & 4 \\
Topics &  Nordmenniutland-sport-langrenn-hjelmeset-slovenia Nordmenniutland-sport-langrenn-hjelmeset-slovenia Nordmenniutland-sport-langrenn-verdenscup-slovenia \\
No. documents & 3 \\
Documents & www.vg.no/sport/ski/langrenn/artikkel.php?artid=581044 www.aftenposten.no/nyheter/sport/skisport/article3431048.ece www.adressa.no/sport/article1423419.ece \\
\hline
\end{tabular}
\end{center}
\caption{A cluster example}
\label{tab:clusterexample}
\end{table}


\begin{table}[htdp]
\footnotesize
%\rowcolors{1}{gray!10}{white}

\begin{center}
\begin{tabular}{|c|c|c|c|}
\hline
Topic overlap &  Fraction of total clusters & Percentage  & accumulated\\ 
\hline
0 -precision & 253 / 457 & 0.554 & 0.554 \\
1 -precision & 2 / 457 & 0.004 & 0.558 \\
2 -precision & 1 / 457 & 0.002 & 0.560\\
3 -precision & 2 / 457 & 0.004 & 0.565\\
4 -precision & 8 / 457 & 0.018 & 0.582\\
5 -precision & 191 / 457 & 0.418 & 1.000\\
\hline
\end{tabular}
\\Total: 457 (of  457)
\end{center}
\caption{Precision of cluster result}
\label{tab:clusterprecision}
\end{table}


\section{Experimental Research}
\label{ExperimentalResearch}

This section will describe scientific experiments in context of algorithm analysis and outline an experiment design to test the hypotheses formulated to test the \CTC algorithm. The experimental design was formed around some of the techniques described in \cite{Bartz-Beielstein2004}, but with significant adjustments. In his article, \citeauthor{Bartz-Beielstein2004} is primarily concerned with the optimization of a particle swarm optimization algorithm. The techniques are non the less relevant as they provide valuable insight into experimental benchmarking of algorithms. The process is summarized in Table~\ref{tab:experimentsequence}.

\begin{table}[htdp]
\footnotesize
%\rowcolors{1}{gray!10}{white}
\begin{center}
\begin{tabular}{|c|p{10cm}|}
\hline
Step & Action\\
\hline
(S-1) & Pre–experimental planning\\
(S-2) & Scientific hypothesis\\
(S-3) & Statistical hypothesis\\
(S-4) & Specification\\
(a) & optimization problem,\\
(b) & constraints,\\
(c) & initialization method,\\
(d) & termination method,\\
(e) & algorithm (important factors),\\ 
(f) & initial experimental design,\\ 
(g) & performance measure\\
(S-5) & Experimentation\\
(S-6) & Statistical modeling of data and prediction\\
(S-7) & Evaluation and visualization\\
(S-8) & Optimization\\
(S-9) & Termination: If the obtained solution is good enough, or the maximum number of iterations has
been reached, go to step (S-11)\\
(S-10) & Design update and go to step (S-5)\\
(S-11) & Rejection/acceptance of the statistical hypothesis\\
(S-12) & Objective interpretation of the results from step (S-11)\\
\hline
\end{tabular}
\caption{``\textit{Sequential approach. This approach combines methods from computational statistics and exploratory data analysis to improve (tune) the performance of direct search algorithms.}'', from \protect \cite[p. 417]{Bartz-Beielstein2004}}
\label{tab:experimentsequence}
\end{center}
\end{table}

\subsection{Pre-experimental planning}
The first step is pre-experimental planning. During the pre-experimental planning tests were performed to find candidate parameters for the algorithm as well as good parameter ranges (see Chapter~\ref{EvaluationTesting}). Additionally the corpus file was processed to get a suitable format for clustering with different parameters. This included marking sentences up with appropriate text types.

\subsection{Scientific Hypotheses}

To investigate whether the optimization algorithm is effective a hypothesis was defined.


% \begin{description}
% 	\item []Optimal parameter set
% 	\begin{description}
% 	\item [\(H1_{1}\)] There is a parameter set \(p_{optimized}\) for the \CTC algorithm that for the ``Klimauken'' corpus give improved results compared to the average performance of 10 random parameter sets.
% 	\item [\(H1_{0}\)] There is no parameter set \(p_{optimized}\) that improves the performance of the \CTC algorithm for the ``klimauken'' corpus compared to the average performance of 10 random prameter sets.
% 	\end{description}
% \end{description}

\begin{description}
	\item []Optimization algorithm
	\begin{description}
	\item [\(H_{1}\)] The genetic algorithm gives values that for the ``klimauken'' corpus give better performance than the average performance of random values.
	% \item [\(H2_{1}\)] The genetic algorithm described in this thesis produce a parameter set \(p_{optimized}\) that for the ``klimauken'' corpus gives better performance than the average performance of ten random parameter sets.
	% \item [\(H2_{0}\)] The genetic algorithm described in this thesis does not produce a parameter set \(p_{optimized}\) that is worse than the default parameter set \(p_{default}\).
	\item [\(H_{0}\)] The genetic algorithm does not give values that for the ``klimauken'' corpus give better performance than the average performance of random values.
	\end{description}
\end{description}

The hypothesis only test the specific implementation of the Genetic Algorithm as implemented in this thesis work. Other optimization algorithms or other implementations of the Genetic Algorithm might perform differently from this one. By constricting the hypothesis to only account for this implementation of the Genetic Algorithm it is possible to formulate a null hypothesis. The results should however indicate whether this optimization approach has any virtue, or if genetic algorithms are wholly unsuitable for this optimization problem.

% \begin{description}
% 	\item []Genetic versus incremental search
% 	\begin{description}
% 	\item [\(H3_{1}\)] The genetic algorithm described in this thesis produce a parameter set \(p_{optimized}\) that is significantly better than an iteratively improved parameter set \(p_{iterativ}\).
% 	\item [\(H3_{0}\)] The genetic algorithm described in this thesis does not produce a parameter set \(p_{optimized}\) that is better than an iteratively improved parameter set \(p_{iterative}\).
% 	\end{description}
% \end{description}

The scope of this thesis does not allow for additional hypotheses to be investigated. For future research an additional possible hypothesis that would be worth testing is whether the optimized parameter set is also optimized for similar corpora.

\subsection{Specification of Experiment}
The specification of an experiment on the optimization of an algorithm should, according to \citeauthor{Bartz-Beielstein2004}, include a specification of the optimization problem, definition of constraints on the optimization, a description of how the algorithm is initialized, and an experiment design which describes the problem to be investigated and the algorithm design.


\subsubsection{Experiment Design}
The algorithm design describes the parameter sets used in the experimentation phase. The algorithm designs are listed in Table~\ref{tab:algorithmdesign}. Each column corresponds to one parameter in the algorithm and is labeled with an abbriviation of the parameter name. Please refer to Table~\ref{tab:parameterabbreviationdictionary} to see a dictionary for the abbriviations.

Three specific algorithm designs have been defined: \citeauthor{Oren1998} parameters, Genetic params, and \citeauthor{Moe2013} parameters. The \citeauthor{Oren1998} parameters are not directly relevant to the experiment. Its inclusion can be argued for because it serves as a interesting comparison for the average performance of the algorithm and the performance with optimized parameters. The \citeauthor{Moe2013} parameters were included on the same basis. These parameters measure how well the algorithm performs on a corpus when the parameters have been manually optimized for the ``Klimauken'' corpus. The last parameter set, Genetic params, were created using the \GA optimization process. It is this parameter set that will be used in the experiments. In addition to these three parameter sets, 100 random parameter sets were created to measure the average performance of the \CTC algorithm on the ``Klimauken'' corpus. This was done to establish a base performance to measure the optimized parameters against. These are not listed in Table~\ref{tab:algorithmdesign} for practical reasons.

The random parameters were generated using the generate\_random\_chromosome function implemented for the genetic algorithm. The spesific values and their limits are listed below:

\begin{description}
  \item[Tree types:] Suffix, Mid-gram, N-gram, Range-gram\hfill
  	\begin{description}
	  \item[Suffix:] No limits
	  \item[Mid-gram:] No limits
	  \item[N-gram:] 1 - 10
	  \item[Range-gram:] min: 0.0 - 0.9 \& max: 0.1 - 1.0
	\end{description}
  \item[Top base clusters amount:] 100 - 10 000
  \item[Min term occurrence:] 5 - 150
  \item[Max term ratio:] 0.05 - 1.00 (rounded to two decimals)
  \item[Min limit base cluster score:] 0 - 20
  \item[Max limit base cluster score:] 3 - 25 (if n > min limit base cluster score, min + 1 if not)
  \item[Drop singleton base clusters:] 0 or 1
  \item[Drop one word clusters:] 0 or 1
  \item[Order descending:] 0 or 1
  \item[Text types:] (0 - 1) for Frontpage Heading, Frontpage Introduction, Article Heading, Article Introduction, Article Byline, Article Text
  \item[Text amount:] 0.0 - 1.0
  \item[Similarity measure:] Etzioni, Jaccard, Cosine, Amendment 1C\hfill
  	\begin{description}
	  \item[Etzioni:] 0.0 - 1.0 (Etzioni threshold)
	  \item[Jaccard:] 0.0 - 1.0 (Jaccard threshold)
	  \item[Cosine:] 0.0 - 1.0 (Jaccard threshold) \& 0.0 - 1.0 (Cosine threshold)
	  \item[Amendment 1C:] 0.0 - 1.0 (Jaccard threshold), 5 - 500 (avg. corpus frequency threshold), \& 0 - 50 (intersect minimum limit) 
	\end{description}
\end{description}


\begin{landscape}
\begin{table}
\begin{center}
	\begin{tabular}{|l|l|l|}
	\hline
	Abbreviation & Full parameter name & Short explanation\\
	\hline
	TreeT & Tree type & The tree type (suffix, mid-gram, etc) used to build the phrase tree.\\
	TBC & Top Base Clusters & The top base clusters limit that determines the amount of base clusters to include.\\
	MinTO & Min term occurrence & The minimal occurrence of term in the corpus for it to not be considered a stop word.\\
	MaxTR & Max term ratio & The max allowed ratio of a term in the corpus for it not to be considered a stop word.\\
	MinLBCS & Min limit base cluster score & The min limit at which the base cluster label is scored the minimum score.\\
	MaxLBCS & Max limit base cluster score & The max limit at which the base cluster label is scored the maximum score.\\
	DSBC & Drop singleton base clusters & Whether to drop singleton base clusters or not.\\
	DOWC & Drop one word clusters & Whether to drop one word clusters or not.\\
	OD & Order descending & Whether to order the base clusters list in descending or ascending order.\\
	TA & Text amount & The amount of article text to include in the snippet expansion phase.\\
	TTy & Text types & Which kinds of text to include in the snippet expansion phase.\\
	SM & Similarity method & Which similarity method to use in the base clusters similarity function.\\
	\hline
	\end{tabular}
	\caption{A dictionary of abbreviations used in Table~\ref{tab:algorithmdesign}}
	\label{tab:parameterabbreviationdictionary}
\end{center}
\end{table}
\end{landscape}

\begin{landscape}
\begin{center}
\footnotesize
  \begin{longtable}{|p{1.5cm}|p{1.5cm}|p{1cm}|p{1.5cm}|p{1.5cm}|p{1.5cm}|p{1.5cm}|p{1.2cm}|p{1.5cm}|p{0.8cm}|p{1cm}|p{2.5cm}|p{1.5cm}|}
  \hline
  Design & TreeT & TBC & MinTO & MaxTR & MinLBCS & MaxLBCS & DSBC & DOWC & OD & TA & TTy & SM\\
  \hline
  \endhead
  \citeauthor{Oren1998} parameters & Suffix & 500 & 4 & 0.4 & 2 & 7 & 0 & 0 & 1 & 0 & Frontpage Introduction, Frontpage Heading, Article Heading, Article Byline, Article Introduction & Etzioni (0.5 threshold)\\
  \hline
  Genetic params & 4-slice & 7937 & 141 & 0.47 & 11 & 17 & False & True & False & 0.32 & Front page intro, Article heading, article intro & Etzioni (0.99) \\
  \hline
  \citeauthor{Moe2013} parameters & Mid-gram & 5000 & 6 & 0.6 & 2 & 7 & 0 & 0 & 1 & 0 & Article Heading, Article Byline, Article Introduction & Etzioni (0.5 threshold)\\
  \hline
    \caption{Algorithm designs used in the experiments.}
	\label{tab:algorithmdesign}
  \end{longtable}
\end{center}
\end{landscape}

In \citetitle{Bartz-Beielstein2004} the problem design describes the optimization problems on which the particle swarm algorithm will work on. One can thus understand problem design as a definition of the problems that the algorithm is to solve. Table~\ref{tab:problemdesign} describes the clustering problem used in the experiment. The Experiment column specify the hypothesis for which the problem design was applied. The algorithm design is a reference to the parameter sets specified in Table~\ref{tab:algorithmdesign}. To keep the results in line with the classifications provided by the experts that prepared the ``Klimauken'' corpus, singleton ground truth clusters were preserved. The F-Beta constant was set to 1 to weigh precision and recall equally.

\begin{table}
\small
\begin{center}
  \begin{tabular}{|l|p{2.5cm}|l|l|l|}
  \hline
  Experiment & Algorithm Design & Drop singelton ground truth & F-Beta constant & Corpus\\ 
  \hline
  E1 & Genetic params & False & 1 & Klimauken\\
  \hline
  E1 & Random 1 & False & 1 & Klimauken\\
  \hline
  E1 & Random 2 & False & 1 & Klimauken\\
  \hline
  E1 & Random ... & False & 1 & Klimauken\\
  \hline
  E1 & Random 100 & False & 1 & Klimauken\\
  \hline
  \end{tabular}
\end{center}
\caption{The problem designs describing the clustering problem for the experiment.}
\label{tab:problemdesign}
\end{table}
%!TEX root = ../Thesis.tex
% Chapter Template

\chapter{Design \& development} % Main chapter title

\label{DesignDevelopment} % Change X to a consecutive number; for referencing this chapter elsewhere, use \ref{ChapterX}

\lhead{Chapter \ref{DesignDevelopment}. \emph{Design \& Development}} % Change X to a consecutive number; this is for the header on each page - perhaps a shortened title

%----------------------------------------------------------------------------------------
% SECTION 1
%----------------------------------------------------------------------------------------
This chapter will briefly touch on the development of the algorithm. The chapter will then go into some detail about the design of the algorithm so that the reader may better understand the code and how the algorithms works. This chapter will also be of benefit to those who wish to use the algorithm or even modify it. The chapter will also explain some of the theory behind the algorithm functions and specify the name of the algorithm's parameters as the clustering process is explained.

\section{Stages of development}
The thesis builds on a project proposal written by \supervisor. The proposal suggests that work be done to optimise the parameters used in the \STC algorithm in order to get better clustering results. The \CTC algorithm written by \supervisor is implemented in Python. In the project draft outlining this thesis work as preparation for the thesis, the project proposal was amended to include the use of a genetic algorithm. This necessitated some modification of the existing implementation. The code is reimplemented in Python 3 and amended to provide easy ways to run specific parameter sets, incremental tests, and a distributed genetic algorithm for optimization.

% I had no familiarity with Python. The \emph{first stage} of development was thus to learn Python and familiarise myself with the implementation of the \CTC algorithm. The algorithm was implemented with largely hard coded parameters with about one separate python module (file) for each parameter configuration. To make testing different parameter sets easier a \emph{second stage} of development was performed. The parameter set were abstracted away so that the clustering function would accept any parameter set. This made it possible to perform initial incremental tests. These incremental tests then informed the design of the genetic algorithm, the \emph{third stage} of development. The genetic algorithm was initially developed to run on a single computer and was designed following the specifications in \cite{Goldberg1989,Negnevitsky2002,Haupt2004a}. Testing of the initial design for the genetic algorithm revealed that it was indeed capable of improving the F-Measure of the \CTC algorithm's results. There were a problem however as testing many chromosomes on a single machine took too much time. It was necessary to distribute the task.

% A \emph{fourth stage} of development was executed to alleviate the problem. A small server - client framework was written to facilitate distribution of computation tasks. The genetic algorithm was then rewritten to send chromosomes (essentially parameter sets) to clients that would then perform the actual clustering tasks. This made it very simple to run the clustering job on multiple CPU cores and multiple computers. Additionally a corpus processor was implemented to convert other corpus files to the snippet format used for the \CTC algorithm. A new subclass of the corpus processor needs to be implemented for each additional corpus to be converted.

% \subsection{Memory handling}
% One problem with Python 2 is the way it handles memory. The client processes would, at least in some environments, consistently eat up as much memory as they could and never free up memory used. This seemed to be a problem related to the way in which Python 2 allocates and deallocates memory for small variables such as integers. Essentially the Python 2 virtual machine would hold on to memory space allocated to small variables in case it could need such space later. This in essence saves CPU cycles as it does not have to ask for more space later. The problem seemed to be that it did not reuse that space later, but instead opted to ask for a new virtual memory space for small variables. Because of this the entire project had to be ported to Python 3.

\section{Subtle variance in test results}
Porting the \CTC algorithm from Python 2 to Python 3 proved an issue. The original \CTC implementation had produced consistent results. After porting the algorithm from Python 2 to Python 3, the algorithm started producing slightly varying results between test runs. It turns out that the problem was caused by the use of dictionaries (\texttt{dict} type). This use was affected by the transition from Python 2 to Python 3. In Python 2 the order in which items (key-value pairs) are inserted into a dictionary is stored implicitly\footnote{\url{http://docs.python.org/2.7/library/stdtypes.html\#dict}}. When you iterate through the items in the dictionary, they are extracted in the same order as they were inserted. This is not true for the Python 3 implementation of dictionaries. In Python 3 you are given a view\footnote{\url{http://docs.python.org/3.3/library/stdtypes.html\#dict}} when you ask for an iterator over the items in the dictionary, and this view does not necessarily give you the items in the same order as they were inserted. The solution was to implement the compact trie subtree dictionaries as \texttt{OrderedDict} \footnote{\url{http://docs.python.org/3.3/library/collections.html\#collections.OrderedDict}} objects to get the same order each time. This ensures that the resulting base clusters are the same on each run of the algorithm and thus secure the consistency of the clustering results.

\section{Overview of system}
This section will give an overview of the algorithms. It will cover the overall design of the algorithms and diverge briefly into some technical aspects. The entire code base of the project is too big to be covered in its entirety. The code is open source and can be found in full at \href{http://github.com/snorremd/distributed-clustering}{GitHub - http://github.com/snorremd/distributed-clustering}. The code is licensed under the MIT license and can be used and modified freely.

\subsection{\CTC}
The original implementation of the \CTC algorithm is authored by \cite{Moe2014}. The implementation used in this thesis work is more or less the same, but with modifications to make it possible to dynamically specify parameter sets. The \CTC algorithm is implemented as a class. The class initialiser (similar to a constructor) takes as parameters a \texttt{Corpus} object, that specifies which corpus is to be used, and a \texttt{ClusterSettings} object that informs the \CTC object if it is to include singleton ground truth clusters and which value to use for the beta value for the F-Measure. Singleton clusters are those clusters that only have a single document in them. Including singleton ground truth clusters may make the genetic algorithm favour solutions with many singleton clusters, but one should be careful about adjusting the ground truths specified in a corpus. The F-Measure's beta value determines the relative weight distribution between the precision and recall values used to calculate the F-Measure score. Using a very low beta value will essentially make the F-Measure a measure of precision. Conversely using a very high beta value will make it a measure of recall. A F-Measure beta value of 1 makes it a harmonious mean of the two. The measure is then called the harmonious F-Measure. The value can be used to tune the kind of results the genetic algorithm optimise for. In this thesis the harmonious F-Measure has been used.

The initialiser continues by building an index of ground truth clusters, a tag-index, and indexes over different frequency measures (corpus frequency, raw frequency and document frequency) for the terms contained within the corpus. The raw frequency of a term is the number of times that term occurs in the corpus. The ground truth index and tag index are both created based on the snippet file specified in the \texttt{Corpus} object. These frequencies are later used for some of the similarity measures. The snippet file is marked up in XML and use the structure shown in Listing~\ref{lst:snippetfile}. Essentially the snippet file encompass a entire corpus. It is divided into snippet elements which correspond to one document. Each snippet is then divided into text type elements which comprise one of six forms of text from that document. Each text type element, for example ``ArticleText'', then holds snip (the actual snippets) elements that are of that text type. Ground truth clusters are then the those documents that have equal tag-values. The tag index is an index of sources (documents) pointing to the tag value of that document. It is built by collecting the source and tag-values of each document.

\begin{lstlisting}[float=ht, language=xml, breaklines=true, label=lst:snippetfile, caption={Snippet file encoded in XML}]
<?xml version='1.0' encoding='ascii'?>
<snippetcollection source="klimaukenOBT.xml">
    <snippet id="2009-12-07-aa-01" source="http://www.adressa.no/nyheter/trondheim/article1419658.ece" tags="Innenriks-ulykker-trafikk-utforkj&#248;ring-trondheim">
        <ArticleIntroduction>
            <snip> bil havne bokstavelig tale hel kant Nedre Elvehavn mandag ettermiddag</snip>
        </ArticleIntroduction>
        <ArticleText>
            <snip> bil havne hel kant Nedre Elvehavn mandag ettermiddag</snip>
            <snip> bil tom</snip>
            <snip> If&#248;lge politi S&#248;r-Tr&#248;ndealg skulle bil tom komme sted</snip>
            <snip> menneske bil politi komme</snip>
            <snip> brannvesen sikre bil Falck rekvirere berging fortelle Curt Ivar R&#248;hmen operasjonsleder S&#248;r-Tr&#248;ndelag politidistrikt</snip>
            <snip> st&#229; fri</snip>
            ...
        </ArticleText>
        <ArticleByline>
            <snip> Tore Lagesen helle bil nesten ende vann Nedre Elvehavn</snip>
        </ArticleByline>
        <ArticleHeading>
            <snip> Biltur hel kant</snip>
        </ArticleHeading>
        <FrontPageIntroduction>
            <snip> En bileier Trondheim flaks side bil ta tur h&#229;nd mandag ettermiddag</snip>
            <snip> le mye</snip>
        </FrontPageIntroduction>
        <FrontPageHeading>
            <snip> telefon bil tur elv</snip>
        </FrontPageHeading>
    </snippet>
    <snippet>
      ...
    </snippet>
    ...
  </snippetcollection>
\end{lstlisting}

The \CTC algorithm is implemented as a method named \texttt{clustering} in the \CTC class. The application applies a \texttt{chromosome}, which essentially act as a parameter set (see Listing~\ref{lst:chromosome}, to the method and then executes each step of the \CTC algorithm according to the parameter values supplied. The method takes the following parameters:
\begin{description}
\item[Tree type]\hfill \\The tree type (suffix, mid-gram, etc) used to build the phrase tree.
\item[Top Base Clusters]\hfill \\The top base clusters limit that determines the amount of base clusters to include.
\item[Min term occurrence]\hfill \\The minimal occurrence of term in the corpus for it to not be considered a stop word.
\item[Max term ratio]\hfill \\The maximal allowed ratio of a term in the corpus for it not to be considered a stop word.
\item[Min limit base cluster score]\hfill \\The min limit at which the base cluster label is scored the minimum score.
\item[Max limit base cluster score]\hfill \\The max limit at which the base cluster label is scored the maximum score.
\item[Order descending]\hfill \\Whether to order the base clusters list in descending or ascending order.
\item[Drop singleton base clusters]\hfill \\Whether to drop singleton base clusters or not.
\item[Drop one word clusters]\hfill \\Whether to drop one word clusters or not.
\item[Text types]\hfill \\Which kinds of text to include in the snippet expansion phase.
\item[Text amount]\hfill \\The amount of article text to include in the snippet expansion phase.
\item[Similarity method]\hfill \\Which similarity method to use in the base clusters similarity function.
\end{description}.

\begin{lstlisting}[float=ht!, language=python, label=lst:chromosome, caption={An example chromosome}]
fitness = 0
id = 1
idCounter = 2
results = ## Ommitted
tree_type = (0,0,0)
top_base_clusters_amount = 992
min_term_occurence_in_collection = 23
max_term_ratio_in_collection = 0.72
min_limit_for_base_cluster_score = 3
max_limit_for_base_cluster_score = 7
should_drop_singleton_base_clusters= 0
should_drop_one_word_clusters = 1
text_amount = 0.73
text_types = {
  "FrontpageIntroduction": 1,
  "FrontpageHeading": 0,
  "ArticleHeading": 1,
  "ArticleByline": 1,
  "ArticleIntroduction": 0,
  "ArticleText": 1
}
similarity_measure = {
  similarity_method: 2,
  params: (0.5, 10, 1)
}
descending_order: 1
\end{lstlisting}

\subsubsection{Snippet filtering}
\label{subsubsec:snippetfiltering}
The first step in the \CTC algorithm is snippet filtering. The algorithm filters the snippet list according to which \emph{text types} should be included. Because the parameter is randomised there are cases where the chromosome specifies that no text types should be included. In such cases the algorithm will return an empty result. An empty result is a result were each performance measure is given a zero score. The chromosome also specify a \emph{text amount} ratio that tells the algorithm how much of the article text to include. This ratio is in the range 0 \dots 1 with a .01 increment. In the event that article text should be included the number of snippets to include are then simply calculated by multiplying the number of article text snippets with the ratio.

\subsubsection{Snippet expansion}
\label{subsubsec:snippetexpansion}
The algorithm then moves on to the snippet expansion phase. It selects the expansion technique given by the \emph{tree type} parameter in the chromosome. Research shows that it is possible to achieve results with n-grams comparable to suffixes, but in less time, \parencite{Moe2014compact}. The expansion technique (tree type) may be one of the following: suffix, n-gram, mid-gram or range-gram expansion. Each expansion technique will be explained with an example. Given a snippet \(S\): ``mouse run trough house order find cheese is discovered cat chase away'', we can define the snippet as \(S = t_{1} \dots t_{12}\). \(S\) can be expanded using each of the four expansion techniques described below.

Each suffix phrase \(P\)  of \(S\) are defined as: \(P = t_{12-m+1} \dots t_{12}\) where \(0 \le m < 12\). This gives us the following suffixes for the snippet:
\begin{inparaenum}[\itshape 1\upshape)]
\item ``mouse run through house order find cheese is discovered cat chase away'',
\item ``run through house order find cheese is discovered cat chase away'',
\item ``through house order find cheese is discovered cat chase away'',
\item ``house order find cheese is discovered cat chase away'',
\item ...,
\item ``chase away'', and
\item ``way''
\end{inparaenum}


A n-gram phrase \(P\) of some fixed length \(n\) is defined as \(P = t_{m} \dots t_{m+n}\) where \(0 \le m \le 12 - n\). This definition gives us the following 6-grams of the snippet:
\begin{inparaenum}[\itshape 1\upshape)]
\item ``mouse run through house order find,''
\item ``run through house order find cheese,''
\item ``through house order find cheese is,''
\item ..., and
\item ``cheese is discovered cat chase away''
\end{inparaenum}

Mid-grams are l-grams where the length \(l\) of the grams is given by the function \(l = round(\frac{phrase length}{2})\). For the example snippet the mid-grams would simply be 6-grams as exemplified above. The last type of expansion is range grams. Range-grams are simply put all the n-grams in the range \(r_{min}, \dots, r_{max}\). \(r_{min}\) is calculated using the function \(min = floor(\text{snippet length} * \text{min~ratio})\). \(r_{max}\) is calculated using a similar function: \(max = ceil(\text{snippet length} * \text{max~ratio})\).  Min and max ratio are values where \(0 < ratio <= 1\). The range grams of the example snippet given the range 0.4 and 0.8 would thus be all n-grams in the range \(4 \dots 10\), i.e. 4-grams, 5-grams, ..., 9-grams, and 10-grams.

\subsubsection{Tree building}
The snippet expansion step returns a list of phrase-source pairs. The algorithm then builds a compact trie over the list by inserting each pair into the trie structure. A simplified diagram of the structure can be seen in Figure~\ref{fig:compacttriedatastructure}. The trie is implemented as a tree structure where each node in the tree is a CompactTrie Node object. The edges in the compact trie are implemented as labels in the node. Thus the edge from the root node to one of the root's child nodes are the label property in that child node.

All nodes have a map (also known as a dictionary or associative array) of their child nodes. The map's keys are the first word in the edge labels which points to the child nodes. The values are the child nodes themselves. Empty subtree maps indicate that a node is a leaf node. All nodes expect the root node have a parent property which connects that node to its parent node. Each node also has a source property which contains a list of all sources that contains the edge label pointing to the node.

\begin{figure}[!ht]
  \begin{center}
    \includegraphics[totalheight=0.3\textheight]{Figures/compacttriedatastructure}
  \end{center}
  \caption{A simplified diagram of the compact trie data structure using two snippets “cat ate cheese” (d1) and “cat ate mouse too” (d2).}
  \label{fig:compacttriedatastructure}
\end{figure}

\subsubsection{Base clusters}
\label{subsubsec:baseclusters}
The \CTC algorithm generates base clusters with a simple recursive function. The function is called and applied to the root node of the compact trie. The function then creates new base clusters for each subtree in the root, and the subtrees of those subtrees etc. When all subtrees have been explored, the function recursively adds the sources of each subtree to their parent base cluster as the function climbs out of the recursion. This way each base cluster's sources (given a node in the compact trie) is the union of all the sources in the descendant nodes.

The algorithm then sorts the base clusters according to their base cluster score in either a descending or ascending order. The order is given by the \emph{Order descending} parameter. \cite{Moe2014} shows that the scoring of base clusters are sensitive to the kind of text that is being clustered. It is shown that ordering the base clusters in an ascending order (i.e. bottom or wort base clusters first) give better results for the ``Klimauken'' corpus. \cite{Oren1998} conversely use a descending order for base clusters (best scores first). The order of the base clusters should therefore be considered a worth while parameter for optimization. After selecting the top base clusters the algorithm either keeps or drops the singleton base clusters, i.e. base clusters that have a single source document.

Recall the function for calculating the effective length of a base cluster label \(f(\vert P \vert)\) where \(f(\vert P \vert) = 0\) for \(\vert P \vert < 2\), \(f(\vert P \vert) = \vert P \vert\) for \(2 \le \vert P \vert < 7\), and \(f(\vert P \vert) = 7\) for \(7 < \vert P \vert\). Here the value 2 and 7 have been replaced by dynamic parameters \emph{min and max limits for base cluster scores}. The min limit is a number in the range 1 \dots 5; the max limit a number in the range 3 \dots 9. If a corpus yields base clusters with generally longer labels, this will allow labels of longer lengths to be scored differently. A higher min limit will reduce the number of base clusters with short labels that might else wise be used in cluster creation.

The effective length of a label is dependent on the document frequency of each word in that label. A word contributes to a labels length if it satisfies two document frequency constraints. It should, according to \cite{Oren1998} have a document frequency of at least 4, and occur in no more than 40\% of the collection's documents. Because word frequencies vary in different corpora the optimal parameters for each of the limits might also vary. Each limit have thus been made dynamic. The \emph{min term occurrence} ranges from 1 to 500, and the \emph{max term ratio} from 0.1 to 1 with increments of 0.01.

\subsubsection{Base cluster merging}
\label{subsubsec:baseclustermerging}
Recall that \cite{Oren1998} merge base clusters using the ratio of common elements in the clusters, to the number of elements in the union of the two base cluster sets. This implementation is fairly naive because it only considers the amount of sources the base clusters share. The implementation produces big clusters with good source overlap (number of shared sources), but runs the risk of producing clusters with low label overlap (few shared label words). This original similarity measure only takes into account whether the two base clusters have some ratio of documents in common. In this thesis and in research by \citeauthor{Moe2014} additional \emph{similarity measures} have been investigated. This thesis will concern itself with three alternative similarity measures.

The first alternative, Jaccard similarity, is based on the Jaccard similarity coefficient which was explained in Chapter~\ref{Theory}. It is quite similar to the Etzioni similarity measure, and as we will later see produce very similar results. Jaccard similarity will not be further discussed here. The second similarity measure, in this text called cosine similarity, incorporates the Vector Space Document model into the \CTC algorithm. Finally a third similarity measure based on the original Etzioni similarity measure was explored. In this measure the frequency of the base cluster labels were included in addition to the document overlap.

The similarity measure based on cosine similarity use the Vector Space Document model and cosine similarity function to find the similarity of base cluster labels. In this context a term's weight (tf-idf) is calculated not using it's document frequencies, but rather it's base cluster frequencies. Given a base cluster \(b\), the concatenation \(S^+\) of \(b\)'s documents, and the corpus collection \(C\) the tf-idf score of a given term \(t\) in \(b\)'s label can be calculated with the function \parencite{Moe2014}:

\begin{displaymath}
\vec{v}(t) = \left\{
  \begin{array}{l l}
    \text{tf-idf}(t, S^+, C) & \quad \text{if $t \in b$}\\
    0 & \quad \text{otherwise}
  \end{array} \right.
\end{displaymath}

Two base cluster labels are said to be similar if they satisfy the condition

\begin{displaymath}
\frac{\sum_{t}\vec{v}(t) * \vec{v}'(t)}
{\sqrt{\sum_{t}\vec{v}(t)^2} * \sqrt{\sum_{t}\vec{v}'(t)^2}}
\ge \theta_{cos}
\end{displaymath}

where \(\theta_{cos}\) is the threshold with which similarity is determined. A \(\theta_{cos}\) close to 1 will prevent base clusters from being merged, whilst a value close to 0 would allow close to all base clusters satisfying Etzioni similarity to be merged. Finding the right \(\theta_{cos}\) should therefore be part of the optimization task.

The third similarity measure, for convenience sake named Amendment similarity, also use the label of base clusters when determining the similarity of two base clusters. Two base clusters are similar iff they are Etzioni similar, the average corpus frequency (\(cf(w)\)) of their label words are below some threshold (\(\theta avg\)), and that they share at minimum (\(\theta min\)) number of label words, \parencite{Moe2014}. The amendment can be expressed mathematically, where \(w\) represent a label word, as:

\begin{displaymath}
\frac{\sum\limits_{w \in b \cup b'} cf(w)}{\vert b \cup b' \vert} < \theta avg \quad \text{and} \quad \vert b \cap b' \vert > \theta min
\end{displaymath}

This similarity measure aims to capture those base cluster pairs who's labels indicate that the merged cluster would be extremely general. That is base cluster pairs who's average corpus frequency are higher than some threshold considered average for the collection. The measure also attempts to filter out those clusters that does not share label words.

Merging the base cluster produces a list of merged base clusters, or components. Each component is essentially a double linked list of base clusters. These components are converted to final clusters by collecting the sources and labels from the base clusters, and additionally measuring the source and label overlap of the cluster. The source overlap is the number of common sources in the base clusters in the component.

After merging base clusters into component clusters, the algorithm can drop or keep one word clusters. This is determined by the \emph{Drop one word clusters} parameter, and is included for reasons explained by \cite[][664]{Moe2014}: ``\textit{There will inevitably be clusters cemented by the co-occurrences of a single common word. Such clusters are often large and inaccurate. A one-word cluster is not necessarily a bad cluster but it seems reasonable to assume that this is the case more often than not.}''.

\subsubsection{Component Merge Implementation}
\label{subsubsec:componentmerge}
In the component merge step two components are merged by computing the union of the set of base clusters in the first component and the set of base clusters in the second component. There are a few ways to implement this with varying time complexities. While the original implementation, described below, worked well for the previously investigated parameters, some testing revealed that some parameter configurations would produce components that would make the implementation extremely slow. It was therefore absolutely necessary to investigate alternative implementations to make extensive parameter tests feasible.

The first implementation that was investigated uses a list implementation to store base clusters in the components, with a naive double for loop to find the union. For each base cluster in the second component the algorithm loops through the list of base clusters in the first component to check if the base cluster exists. If this is the case, the algorithm can store the common base cluster. The worst case time complexity of this algorithm is \(O(n*m)\) where \(n\) is the number of base clusters in the second component, and \(m\) the number in the first component. In some cases \(n\) and \(m\) can be almost equally big thus producing a quadratic time complexity. What this in practice means is that some parameter sets can have merge steps running for hours on end for very big base cluster amounts and low merge thresholds.

To solve this problem two additional implementations were investigated. The first attempt at a solution use simultaneous iteration over the two loops. This is done by first sorting the lists of base clusters (an at worst \(O(n \log n)\) operation). When the lists are sorted (by the base clusters' ids) one can iterate through the lists simultaneously and make sure that there is no id from component 1 that is equal to the current id from component 2. This yields a time complexity of \(O(m)\) or \(O(n)\) depending on the length of the lists. This implementation thus works very well for large values of both \(n\) and \(m\). For small values of \(n\) and large values of \(m\) it is still a good deal faster than the \(O(n*m)\) implementation, but not fast enough.

\begin{lstlisting}[float=ht, language=python, breaklines=true, label=lst:simultaneousmerge, caption={Simultaneous merge of components.}] 
base_clusters_2 = list(component2.base_clusters)
base_clusters_1 = list(component1.base_clusters)
list.sort(base_clusters_1, key=lambda bc_tuple: bc_tuple[0])
list.sort(base_clusters_2, key=lambda bc_tuple: bc_tuple[0])

i = 0  # base_clusters_1 counter
j = 0  # base_clusters_2 counter
base_clusters_add = []

while j < len(base_clusters_2):
  if i < len(base_clusters_1):
    id_1 = base_clusters_1[i][0]
    id_2 = base_clusters_2[j][0]

    if id_2 < id_1:  # This base cluster is not in base_clusters_1
      base_clusters_add.append(base_clusters_2[j])
      j += 1
    elif id_2 > id_1:  # There might be a base cluster in base_clusters_1
      i += 1
  else:  # They are the same, iterate both lists
    i += 1
    j += 1
else:  # No more base_clusters_1, add rest of two
  if id_2 != id_1:
    base_clusters_add.append(base_clusters_2[j])
    j += 1

for base_cluster in base_clusters_add:
  heappush(component1.base_clusters, base_cluster)
\end{lstlisting}
   
The second attempt at a solution use the dictionary class in Python to achieve lower time complexities. Each component implements its set of base cluster as a dictionary where the key is the base cluster's id (an integer) and the value is the base cluster object itself. The worst case time complexity of the dict object is linear time, \(O(n)\) for insert and get operations.\footnote{\href{https://wiki.python.org/moin/TimeComplexity}{TimeComplexity - Python Wiki}}. If the dict object consistently performed at worst case time complexity the run time of the algorithm would be the same as with the old list based implementation. Because the algorithm has no run time requirement for each base cluster insertion or base cluster check the worst case time complexity is not very important. Instead we can look at the average case time complexity to determine the suitability of the data structure. The average time complexity for both the insert and the get operation in the dict object is \(O(1)\), considerably better than that of default lists. The time complexity of inserting the base clusters into the component would thus be an \(O(n)\) rather than an \(O(n \log n)\) operation. The merging operation would have a guaranteed time complexity of (on average) \(O(n)\). This ensures that the merge process runs fast even for large \(m\) and small \(n\) values.

\subsection{Genetic Algorithm}

\subsubsection{Genetic algorithm}
As previously written the genetic algorithm is designed to follow the specifications in \cite{Goldberg1989,Negnevitsky2002,Haupt2004a}. The genetic algorithm itself is designed as a class and has a constructor which accepts configurable settings for things like population size, mutation rate, selection rate (keep size) and selection type. Additionally it requires a database handler with which it can store results. The genetic algorithm constructor generates an initial population of the specified size.

The evolution stage have been divided into a few operations. First the algorithm calculates generational data. This includes the average fitness of the chromosomes, average numbers for the different performance measures namely: precision, recall, F-Measure, ground truth and ground truth represented. The average scores and scores held within the top, bottom and median chromosomes are all stored in a MySQL database.

After that comes the ``generation step'' which performs the evolution itself. It starts by discarding the bottom chromosomes given by the keep size. The remaining chromosomes are then used for mating. The algorithm only implements the roulette wheel selection type. Offspring are produced by slicing the parameter set in the two parent chromosomes and then combining the first and last halves respectively to form two new chromosomes.

After creating the offspring a random selection of the chromosomes in the entire population are mutated. The mutation rate and number of genes determine the number of mutations that occur in the population. The last step takes the offspring chromosomes and those chromosomes that were mutated and sends them to the task organiser which is responsible for creating execution tasks and sending them to the clustering algorithm clients.

\subsubsection{Chromosome}
The chromosome class models a chromosome in the genetic algorithm. It contains a number of genes (see Listing~\ref{lst:chromosome}) which also act as the parameter set for the \CTC algorithm. The constructor of the Chromosome class takes each parameter as a constructor parameter. The ``calc\_fitness\_score'' method applies itself as an argument to a cluster method in a ``CompactTrieClusterer'' object. In this way it sends the \CTC parameters to the clustering algorithm. The cluster method in the ``CompactTrieClusterer'' object in turn insert the clustering result into the chromosomes result property. The fitness function then use the top two F-Measure scores to calculate its own fitness. See Listing~\ref{lst:chromosomefitness} for details.

\begin{lstlisting}[float=ht, language=python, breaklines=true, label=lst:chromosomefitness, caption={Fitness function in the Chromosome class.}]
def calc_fitness_score(self, compact_trie_clusterer):
    """
    Returns the fitness of the chromosome

    Args:
        cSetting (clusterSettings): An object wrapping data
        and settings needed to run the clustering algorithm,
        the parameters in chromosome excluded.

    Calculate fitness as the average of the two
    """

    self.result = compact_trie_clusterer.cluster(self)
    fMeasure0 = self.result.f_measures[0]
    fMeasure1 = self.result.f_measures[1]
    self.fitness = fMeasure0 + fMeasure1
\end{lstlisting}

The chromosome module additionally implement a cross over function for producing offspring. This can be seen in Listing~\ref{lst:crosschromosomes}. It takes two chromosomes and retrieves their genes as tuple objects. It then crosses them by calculating a random cross over point and slicing the tuples at that position. The function then calls the ``genesTupleToChromosome'' function to create two new chromosomes for the new genes.

\begin{lstlisting}[float=ht, language=python, breaklines=true, label=lst:crosschromosomes, caption={Code for crossing chromosomes}]
def crossChromosomes(chromosome1, chromosome2):
    """
    Takes two  parent chromosomes cross them and return two children
    chromosomes
    """
    genes1 = chromosome1.genesAsTuple()
    genes2 = chromosome2.genesAsTuple()
    crossOverPoint = randint(1, len(genes1) - 1)
    genes12 = genes1[0:crossOverPoint] + genes2[crossOverPoint:len(genes2)]
    genes21 = genes2[0:crossOverPoint] + genes1[crossOverPoint:len(genes1)]

    return [genesTupleToChromosome(genes12),
            genesTupleToChromosome(genes21)]
\end{lstlisting}

\subsection{Distribution framework}
The \CTC algorithm performs rather slowly because of its \(O(n^2)\) base cluster merging step. The problem grows quadratically for n number of base clusters. To make larger optimization tasks feasible a distribution framework was implemented. The distribution framework also has the added benefit of making it easy to utilise all CPU cores by running one client for each one. Chapter~\ref{EvaluationTesting} will show how it is feasible to run the optimization on a single computer. This information became apparent only after testing had been done.

The framework consists of two parts: a server, and a client. The server handles all communication between itself, the clients and the genetic algorithm. It consists of the following main classes: Server, ClientHandler, TaskOrganizer, and a ScoreBoard. The server instance handles incoming connections and creates ClientHandler instances to communicate with the clients. Very basic authentication of clients are performed before the client is permitted to join the computation task. A client instance will send a task request message to the client handler. Upon acquisition of the message the client handler will ask the task organiser for available clustering tasks. If any tasks are available it will send these to the client handler. The client will then use a \texttt{CompactTrieClusterer} object to perform the clustering task and send the results back to the client handler.

The task organiser keeps a timeout index of all tasks given to a client handler. If a task has been in this index for a certain amount of time, the task will be put back into the list of available tasks. This ensures that tasks that never would have been finished, as a result of connection issues crashes, can be given to other clients.

Once all the tasks have been completed the task organiser notifies the genetic algorithm. The genetic algorithm then initiates its evolution method. When the genetic algorithm has finished creating a new generation of offspring, it creates new clustering tasks which is added to the list of available tasks. This cycle continues until the genetic algorithm finishes the last generation or meets a cutoff criteria. See the source code for a detailed look at how the interaction occurs.
% Chapter Template

\chapter{Analysis and Discussion} % Main chapter title

\label{AnalysisAndDiscussion}

\lhead{Chapter \ref{AnalysisAndDiscussion}. \emph{Analysis and Discussion}}


Chapter introduction. Should here provide some information about which parts of the work are going to be discussed.
Should talk about the test results and how they correspond to the hypotheses. I.e. Does the testing reveal that a
better parameter set has been found than the default one. Does this parameter set perform better than the default
in different corpora? (Should perhaps test on two additional corpora?). This chapter should also investigate wether
the test results are statistically significant (can I say yes or no on the null hypotheses?). Give definitive or
estimated answer pending test results.

\section{Results}
\label{Results}
Summarize and discuss results.

\section{Validity and relevance}
\label{ValidityRelevance}
Show that data gathered are both valid and relevant. I.e. is the method of research rigorous and correct (methods of data gathering and testing).

And does the data answer the hypotheses. Also discuss the statistical significance of the data in relation to hypotheses.

\subsection{Data autenticity}
Discuss how the validity of data should not be an issue even though the algorithm is distributed. (I.e. results from clients are validated).
Algorithm deterministic...

\subsection{Effects of two different measurements}
Discuss how the varying measurements might affect the results... Does using one measurement over the
other invalidate results? Should both be used (one to measure single category documents, the other to 
measure multiple category documents)?


\subsection{Acceptance/Rejectance test}
Here I will accept or reject the hypotheses based on the discussion of the results above.
\chapter{Summary and Conclusion} % Main chapter title
\label{Conclusion} % Change X to a consecutive number; for referencing this chapter elsewhere, use \ref{ChapterX}
\lhead{Chapter \ref{Conclusion}. \emph{Conclusion}} % Change X to a consecutive number; this is for the header on each page - perhaps a shortened title
Before starting the work with this thesis I saw that there was a need for a system that would lower the barrier of
entry to added semantic metadata to Web pages.
I believe that having a Web of linked semantic data will become increasing important to find relevant information as the amount
of information on the Internet continues to grow.

The advent of schema.org has made it easier to embed metadata on Web pages.
The scope of the ontology however is largely limited to commercial and search engine specific concepts.
In addition to this the microdata format pushed by the authors of schema.org does not allow for mixing vocabularies.
I wanted to improve the situation by making it possible to add metadata about arbitrary content.
My goal when starting work with this thesis was to create a prototype of a system that would allow users to add metadata to Web pages by using natural language,
without requiring the to know the formal underpinnings of ontologies.

A research question was formalized that stated:

\emph{"Is it possible to create a tool which allows naïve users to easily add metadata to their Web sites using natural language?"}

To answer this question there a number of sub-questions that has to be answered.
How should users pick the parts of a Web page they want to add metadata to, and find the concepts it describes using natural language?
Is WordNet suitable for representing disambiguated concepts from natural language in a way that will allow us to map these concepts to formal ontologies?
How should an algorithm be implement to finds mappings from the natural language concept to types in
formal ontologies in a way that preserves the semantic content of the concept?
Is it possible to add metadata to Web pages in such a way that it does not change the way the page is rendered by browsers?

In this thesis I attempted to use WordNet as a method of representing natural language.
WordNet was developed to have word boundaries corresponding to how humans mentally represent concepts.
Using WordNet also made it possible to  build on earlier work on creating mappings between WordNet and formal ontologies for the Semantic Web.
WordNet also contains the concept of the hypernym, a relation that could be utilized to find mappings to higher level concepts
if the synset itself did not contain a direct mapping to an ontology.

I examined the two algorithms for finding the best mapping from a given synset into the ontologies the system was mapping to.
These were later evaluated to find which of them gave the best mappings.

A Web application was created to lets users add metadata to Web pages by selecting content on the page,
and disambiguating the selection by clicking on suggested interpretations of the selection.
\Theartefact\ also allows users to import and export Web pages into the Web application for mark up.

\section{Findings}
I will now summarize the results of the analysis that was done in chapter \ref{AnalysisAndDiscussion}.

The results found to a large degree supports the feasibility of using WordNet as a way to represent natural language
when mapping to formal ontologies.
Some cases where the integrity of the hypernym relation was violated were found.
These errors have been reported to the maintainers of WordNet, and they will be fixed in the next public  release (C. Fellbaum, personal communication, May 1, 2013 and May 13, 2013).
I found that WordNet was unable to capture the grammatical number of natural language.
This means that the tool will not have any way to distinguish between ontological types that differ in this respect.
There were only a few instances in the ontologies that were utilized in the thesis of this flaw hindering the completeness of the mapping.

It was found that using hypernyms as a basis for mapping gave correct mappings.
The incorrect mappings that were discovered during analysis were found to be caused by errors outside of the system.
One should continue to look for further discrepancies in the future to help the development of these tools as well.
The hypernyms first approach does however result in high-level mappings, and could benefit from further refinement.

The metadata that was created using \theartefact\ is comparable to that present at current Web sites which use schema.org to
enrich their content.
The testing also showed that the tool could help users avoid using incorrect types and properties,
since it limits the properties allowed to add to those that are defined as belonging to the type.

As described in section \ref{Rendering} the testing of how the Web pages were rendered after metadata was added by the tool
showed that the documents were displayed in the same manner before and after metadata was added.
My analysis did uncover cases in which adding metadata could potentially change how the page was displayed.
The tests did however find that the system was able to add metadata to the Web pages without changing the way they were displayed in the browser.


\section{Further work}
The goal of this thesis has been to create a functioning prototype of an artefact that allows users to add metadata
to Web pages by using natural language.
The system has been able to fulfill the most basic requirements, and shown that the concept is feasible.
There are now several new interesting ways the tool could be developed further to increase its value to users,
and to researchers.

It would be interesting have mappings to more ontologies,
and one could offer the user a chance to say what the topic of the page was.
In this way one could offer mappings to the Friend Of A Friend ontology if it was a Web page dealing with
social interaction, the Good Relations ontology if it was a commerce page and so on.
To do this the mapping module should get further development to complete the process of uncoupling the ontologies
from the code.

There should be developed a way to allow for multiples of a single property on the Web page.
The idea of adding properties came quite late in the project,
and is as a consequence not as feature rich as it should be.
One should also research into finding some way of allowing for properties from other ontologies.
One difficulty here would be finding a way of presenting these without overwhelming the user.

Adding multiples might also mitigate the issue that synsets are not regarded as distinct because of their grammatical number.
For schema.org the issue of aggregated terms is limited as it only has two types that are the aggregation of multiples of a type.
By allowing multiples of properties, the system could handle adding the aggregation of these behind to the document automatically.
This would be a good solution for the issue with schema.org,
but it might not scale well if the system is expanded to include other ontologies that separate concepts by way of grammatical number.

When the Web site has experienced more usage it would be exciting to examine the usage logs to see which types of text gets tagged.
Actual usage data would be an interesting source to discover concepts that users frequently want to map.
Examination of these logs could therefore be a useful starting point to find out which ontologies to create mappings for,
and which parts of these ontologies which would create the most value for the users.
The usage data collected will make it possible to extrapolate text that the system did not find good disambiguations for
by assuming that this is text that was selected but where the user did not click any of the suggested senses.
It is also possible to count the frequencies at which different synsets or DBPedia terms were chosen as the concept the
user wanted to describe, and use this to target the work of mapping.

\section{Conclusion}
My goal in this thesis was to answer the question if one could create a tool that let users add metadata to a Web page using
natural language.
I have now described the process of developing the prototype of \theartefact\ and the testing and analysis of the results.
The findings have produced positive results for the main research questions.
I have found a representational language that can capture the central semantics of natural language,
and a means of mapping this language to ontologies with the help of mapping files.
I have also managed to add the metadata to the Web pages without changing the rendering of the Web pages.

I acknowledge that there is still a lot of work that needs to be done before the system is finished.
The system has traded expressiveness for simplicity, both to make it easier to use and to make the scope of the project manageable.
The prototype has however been capable of answering my research question,
and I feel confident that it has demonstrated the feasibility of creating a system that creates semantic metadata by
utilizing natural language.


%----------------------------------------------------------------------------------------
%	THESIS CONTENT - APPENDICES
%----------------------------------------------------------------------------------------

%\addtocontents{toc}{\vspace{2em}} % Add a gap in the Contents, for aesthetics

%\appendix % Cue to tell LaTeX that the following 'chapters' are Appendices

% Include the appendices of the thesis as separate files from the Appendices folder
% Uncomment the lines as you write the Appendices

%%!TEX root = ../Thesis.tex
% Appendix A

\chapter{Incremental test results} % Main appendix title

\label{AppendixA} % For referencing this appendix elsewhere, use \ref{AppendixA}

\lhead{Appendix A. \emph{Incremental Test Results}} % This is for the header on each page - perhaps a shortened title

This appendix contains diagrams of all incremental test results. It serves as a reference for discussion and descriptions in section~\ref{Testing}. The appendix is divided into two main sections. The first contains results from incremental tests using the parameters specified by Etzioni and Oren as base values. The second section contains results using the parameters suggested by \supervisor as base values.

\section{Oren \& Etzioni parameters}

The following results were obtained by applying the parameters used by Etzioni and Oren as base parameters for the incremental tests. The parameters were collected as closely as possible, but the types of text collected might not be a complete match as different corpora are used.

\begin{lstlisting}[float=t, language=python, label=lst:etzioniparams, caption={Parameter set used in Oren and Etzioni.}]
<parameters>
    <tree_type>(1,0,0)</tree_type> <!-- Suffix tree -->
    <top_base_clusters_amount>500</top_base_clusters_amount>
    <min_term_occurrence_in_collection>4</min_term_occurrence_in_collection>
    <max_term_ratio_in_collection>0.4</max_term_ratio_in_collection>
    <min_limit_for_base_cluster_score>2</min_limit_for_base_cluster_score>
    <max_limit_for_base_cluster_score>7</max_limit_for_base_cluster_score>
    <sort_descending>1</sort_descending>
    <should_drop_singleton_base_clusters>0</should_drop_singleton_base_clusters>
    <should_drop_one_word_clusters>0</should_drop_one_word_clusters>
    <text_amount>0</text_amount>
    <text_types>
    {
    	"FrontPageIntroduction": 1, "FrontPageHeading": 1,
    	"ArticleHeading": 1, "ArticleByline": 1,
    	"ArticleIntroduction": 1, "ArticleText": 0
    }
    </text_types>
    <similarity_measure>
    {
    	"similarity_method": 0,
    	"params": (0.5, 0, 0)
    }
   	</similarity_measure>
</parameters>
\end{lstlisting}

%!TEX root = ../Thesis.tex
% Incremental tests using Oren & Etzioni's parameters as base


\setstretch{1}
% Define bar chart colors
\definecolor{bblue}{HTML}{3366CC}
\definecolor{rred}{HTML}{DC3912}
\definecolor{ggreen}{HTML}{109618}

% TREE TYPE TESTS
\begin{diagram}[H]
  \begin{center}
\begin{tikzpicture}
  \begin{axis}[
    width  = 0.8*\textwidth,
    height = 4.55cm,
    % major x tick style = transparent,
    ybar=2*\pgflinewidth,
    ymajorgrids = true,
    ylabel = {Score},
    xlabel = {Tree types},
    symbolic x coords={Suffix,Midslice,Rangeslice 0.1-1.0,5-slice},
    xtick = data,
    scaled y ticks = false,
    enlarge x limits=0.15,
    ymin=0,
    nodes near coords,
    nodes near coords align={horizontal},
    every node near coord/.append style={font=\tiny,rotate=90,color=black,anchor=west,/pgf/number format/fixed},
    enlarge y limits={upper,value=0.5},
    legend cell align=left,
    legend style={
      cells={anchor=east},
      legend pos=outer north east
    }
  ]

  \addplot [style={rred,fill=rred,mark=none},postaction={pattern=north east lines,pattern color=white}] table [col sep=semicolon,y=Ground Truth 0] {Diagrams/Etzioni/testTrees.csv};
  \addplot [style={bblue,fill=bblue,mark=none},postaction={pattern=north west lines,pattern color=white}] table [col sep=semicolon,y=Ground Truth Rep 0] {Diagrams/Etzioni/testTrees.csv};
  \addplot [style={ggreen,fill=ggreen,mark=none},postaction={pattern=horizontal lines,pattern color=white}] table [col sep=semicolon,y=fMeasure 0] {Diagrams/Etzioni/testTrees.csv};

  \legend{Ground truth 0,Ground truth rep 0, F-Measure 0}
  
  \end{axis}
\end{tikzpicture}
  \end{center}
  \caption{Performance of the \CTC algorithm for different expansion techniques.}
  \label{diag:treetypesetzioni}
\end{diagram}

% N-SLICE
\begin{diagram}[H]
  \begin{center}
\begin{tikzpicture}
  \begin{axis}[
    width  = 0.8*\textwidth,
    height = 4.55cm,
    % major x tick style = transparent,
    xlabel = {N-Slice length},
    ylabel = {Score},
    %xtick = data,
    ymin=0,
    legend cell align=left,
    legend style={
      cells={anchor=east},
      legend pos=outer north east
    }
  ]

  \addplot+ [style={rred,mark size=1.5}] table [col sep=semicolon,y=Ground Truth 0,x=n-slice] {Diagrams/Etzioni/testNSlices.csv};
  \addplot+ [style={bblue,mark size=1.5}] table [col sep=semicolon,y=Ground Truth Rep 0,x=n-slice] {Diagrams/Etzioni/testNSlices.csv};
  \addplot+ [style={ggreen,mark=triangle*,mark size=1.5}] table [col sep=semicolon,y=fMeasure 0,x=n-slice] {Diagrams/Etzioni/testNSlices.csv};

  \legend{Ground truth 0,Ground truth rep 0, F-Measure 0}
  
  \end{axis}
\end{tikzpicture}
  \end{center}
  \caption{Performance results of the \CTC algorithm for different lengths of n-slice expansion.}
  \label{diag:nslicesetzioni}
\end{diagram}

% RANGE SLICE TEST
\begin{diagram}[H]
  \begin{center}
\begin{tikzpicture}
  \begin{axis}[
    width  = 0.8*\textwidth,
    height = 4.55cm,
    % major x tick style = transparent,
    ybar=2*\pgflinewidth,
    bar width=5pt,
    ymajorgrids = true,
    ylabel = {Score},
    xlabel = {Range slice length},
    symbolic x coords={0.0-1.0,0.1-0.9,0.2-0.8,0.3-0.7,0.4-0.6,0.5-0.5},
    xtick = data,
    scaled y ticks = false,
    enlarge x limits=0.10,
    ymin=0,
    ymax=0.25,
    nodes near coords,
    nodes near coords align={horizontal},
    every node near coord/.append style={font=\tiny,rotate=90,color=black,anchor=west,/pgf/number format/fixed},
    enlarge y limits={upper,value=0.5},
    legend cell align=left,
    legend style={
      cells={anchor=east},
      legend pos=outer north east
    }
  ]

  \addplot [style={rred,fill=rred,mark=none},postaction={pattern=north east lines,pattern color=white}] table [col sep=semicolon,y=Ground Truth 0] {Diagrams/Etzioni/testRangeSlices.csv};
  \addplot [style={bblue,fill=bblue,mark=none},postaction={pattern=north west lines,pattern color=white}] table [col sep=semicolon,y=Ground Truth Rep 0] {Diagrams/Etzioni/testRangeSlices.csv};
  \addplot [style={ggreen,fill=ggreen,mark=none},postaction={pattern=horizontal lines,pattern color=white}] table [col sep=semicolon,y=fMeasure 0] {Diagrams/Etzioni/testRangeSlices.csv};

  \legend{Ground truth 0,Ground truth rep 0, F-Measure 0}
  
  \end{axis}
\end{tikzpicture}
  \end{center}
  \caption{Performance of the \CTC algorithm for different ranges of range-slice expansion.}
  \label{diag:rangelicesetzioni}
\end{diagram}

% NUMBER OF TOP BASE CLUSTERS
\begin{diagram}[H]
  \begin{center}
\begin{tikzpicture}
  \begin{semilogxaxis}[
    width  = 0.8*\textwidth,
    height = 4.55cm,
    % major x tick style = transparent,
    xlabel = {Base cluster amount},
    ylabel = {Score},
    ymin=0,
    xmin=0,
    legend cell align=left,
    legend style={
      cells={anchor=east},
      legend pos=outer north east
    }
  ]
  \addplot+ [style={rred,mark size=1.5}] table [col sep=semicolon,y=Ground Truth 0,x=Basecluster-amount] {Diagrams/Etzioni/testBaseClusterAmounts.csv};
  \addplot+ [style={bblue,mark size=1.5}] table [col sep=semicolon,y=Ground Truth Rep 0,x=Basecluster-amount] {Diagrams/Etzioni/testBaseClusterAmounts.csv};
  \addplot+ [style={ggreen,mark=triangle*,mark size=1.5}] table [col sep=semicolon,y=fMeasure 0,x=Basecluster-amount] {Diagrams/Etzioni/testBaseClusterAmounts.csv};

  \legend{Ground truth 0,Ground truth rep 0, F-Measure 0}
  
  \end{semilogxaxis}
\end{tikzpicture}
  \end{center}
  \caption{Performance of the \CTC algorithm for different limits on top base clusters amount.}
  \label{diag:topbaseclustersetzioni}
\end{diagram}

% MIN TERM OCCURRENCE
\begin{diagram}[H]
  \begin{center}
\begin{tikzpicture}
  \begin{semilogxaxis}[
    width  = 0.8*\textwidth,
    height = 4.55cm,
    % major x tick style = transparent,
    xlabel = {Min term occurrence},
    ylabel = {Score},
    %xtick = data,
    % ymin=0,
    % xmin=0,
    % xmax=200,
    legend cell align=left,
    legend style={
      cells={anchor=east},
      legend pos=outer north east
    }
  ]

  \addplot+ [style={rred,mark size=1.5}] table [col sep=semicolon,y=Ground Truth 0,x=Min Term Occurrence] {Diagrams/Etzioni/testMinTermOccurrence.csv};
  \addplot+ [style={bblue,mark size=1.5}] table [col sep=semicolon,y=Ground Truth Rep 0,x=Min Term Occurrence] {Diagrams/Etzioni/testMinTermOccurrence.csv};
  \addplot+ [style={ggreen,mark=triangle*,mark size=1.5}] table [col sep=semicolon,y=fMeasure 0,x=Min Term Occurrence] {Diagrams/Etzioni/testMinTermOccurrence.csv};

  \legend{Ground truth 0,Ground truth rep 0, F-Measure 0}
  
  \end{semilogxaxis}
\end{tikzpicture}
  \end{center}
  \caption{Performance of the \CTC algorithm for different limits on minimal term occurrence in collection.}
  \label{diag:mintermoccurrenceetzioni}
\end{diagram}

% MAX TERM RATIO
\begin{diagram}[H]
  \begin{center}
\begin{tikzpicture}
  \begin{axis}[
    width  = 0.8*\textwidth,
    height = 4.55cm,
    % major x tick style = transparent,
    xlabel = {Max term ratio},
    ylabel = {Score},
    %xtick = data,
    ymin=0,
    xmin=0.1,
    xmax=1,
    legend cell align=left,
    legend style={
      cells={anchor=east},
      legend pos=outer north east
    }
  ]

  \addplot+ [style={rred,mark size=1.5}] table [col sep=semicolon,y=Ground Truth 0,x=Max Term Ratio] {Diagrams/Etzioni/testMaxTermRatio.csv};
  \addplot+ [style={bblue,mark size=1.5}] table [col sep=semicolon,y=Ground Truth Rep 0,x=Max Term Ratio] {Diagrams/Etzioni/testMaxTermRatio.csv};
  \addplot+ [style={ggreen,mark=triangle*,mark size=1.5}] table [col sep=semicolon,y=fMeasure 0,x=Max Term Ratio] {Diagrams/Etzioni/testMaxTermRatio.csv};

  \legend{Ground truth 0,Ground truth rep 0, F-Measure 0}
  
  \end{axis}
\end{tikzpicture}
  \end{center}
  \caption{Performance of the \CTC algorithm for different limits on max term ratio in collection.}
  \label{diag:maxtermratioetzioni}
\end{diagram}

% Min Limit BC Score
\begin{diagram}[H]
  \begin{center}
\begin{tikzpicture}
  \begin{axis}[
    width  = 0.8*\textwidth,
    height = 4.55cm,
    % major x tick style = transparent,
    xlabel = {Min Limit BC Score},
    ylabel = {Score},
    %xtick = data,
    ymin=0,
    xmin=0,
    xmax=15,
    legend cell align=left,
    legend style={
      cells={anchor=east},
      legend pos=outer north east
    }
  ]

  \addplot+ [style={rred,mark size=1.5}] table [col sep=semicolon,y=Ground Truth 0,x=Min Limit] {Diagrams/Etzioni/testMinLimitBC.csv};
  \addplot+ [style={bblue,mark size=1.5}] table [col sep=semicolon,y=Ground Truth Rep 0,x=Min Limit] {Diagrams/Etzioni/testMinLimitBC.csv};
  \addplot+ [style={ggreen,mark=triangle*,mark size=1.5}] table [col sep=semicolon,y=fMeasure 0,x=Min Limit] {Diagrams/Etzioni/testMinLimitBC.csv};

  \legend{Ground truth 0,Ground truth rep 0, F-Measure 0}
  
  \end{axis}
\end{tikzpicture}
  \end{center}
  \caption{Performance of the \CTC algorithm for different min limit values for base cluster score with unbounded max limit (max limit = length of longest label).}
  \label{diag:minlimitbcscoreetzioni}
\end{diagram}

% Max Limit BC Score
\begin{diagram}[H]
  \begin{center}
\begin{tikzpicture}
  \begin{axis}[
    width  = 0.8*\textwidth,
    height = 4.55cm,
    % major x tick style = transparent,
    xlabel = {Max Limit BC Score},
    ylabel = {Score},
    xtick = data,
    ymin=0,
    xmin=3,
    xmax=15,
    legend cell align=left,
    legend style={
      cells={anchor=east},
      legend pos=outer north east
    }
  ]

  \addplot+ [style={rred,mark size=1.5}] table [col sep=semicolon,y=Ground Truth 0,x=Max Limit] {Diagrams/Etzioni/testMaxLimitBC.csv};
  \addplot+ [style={bblue,mark size=1.5}] table [col sep=semicolon,y=Ground Truth Rep 0,x=Max Limit] {Diagrams/Etzioni/testMaxLimitBC.csv};
  \addplot+ [style={ggreen,mark=triangle*,mark size=1.5}] table [col sep=semicolon,y=fMeasure 0,x=Max Limit] {Diagrams/Etzioni/testMaxLimitBC.csv};

  \legend{Ground truth 0,Ground truth rep 0, F-Measure 0}
  
  \end{axis}
\end{tikzpicture}
  \end{center}
  \caption{Performance of the \CTC algorithm for different max limit values for base cluster score. Min limit set to \protect\citeauthor{Oren1998} default.}
  \label{diag:maxlimitbcscoreetzioni}
\end{diagram}

% Max Limit BC Score best min value
\begin{diagram}[H]
  \begin{center}
\begin{tikzpicture}
  \begin{axis}[
    width  = 0.8*\textwidth,
    height = 4.55cm,
    % major x tick style = transparent,
    xlabel = {Max Limit BC Score},
    ylabel = {Score},
    xtick = data,
    ymin=0,
    xmin=9,
    xmax=15,
    legend cell align=left,
    legend style={
      cells={anchor=east},
      legend pos=outer north east
    }
  ]

  \addplot+ [style={rred,mark size=1.5}] table [col sep=semicolon,y=Ground Truth 0,x=Max Limit] {Diagrams/Etzioni/testMaxLimitBC2.csv};
  \addplot+ [style={bblue,mark size=1.5}] table [col sep=semicolon,y=Ground Truth Rep 0,x=Max Limit] {Diagrams/Etzioni/testMaxLimitBC2.csv};
  \addplot+ [style={ggreen,mark=triangle*,mark size=1.5}] table [col sep=semicolon,y=fMeasure 0,x=Max Limit] {Diagrams/Etzioni/testMaxLimitBC2.csv};

  \legend{Ground truth 0,Ground truth rep 0, F-Measure 0}
  
  \end{axis}
\end{tikzpicture}
  \end{center}
  \caption{Performance of the \CTC algorithm for different max limit values for base cluster score. Min limit set to 8, the best min limit from incremental test on min limit for base cluster score.}
  \label{diag:maxlimitbcscoreetzioni2}
\end{diagram}

% Drop singleton bc test
\begin{diagram}[H]
  \begin{center}
\begin{tikzpicture}
  \begin{axis}[
    width  = 0.8*\textwidth,
    height = 4.55cm,
    % major x tick style = transparent,
    ybar=2*\pgflinewidth,
    bar width=8pt,
    ymajorgrids = true,
    ylabel = {Score},
    xlabel = {Drop singleton base clusters?},
    symbolic x coords={0,1},
    xtick = data,
    scaled y ticks = false,
    enlarge x limits=0.25,
    ymin=0,
    nodes near coords,
    nodes near coords align={horizontal},
    every node near coord/.append style={font=\tiny,rotate=90,color=black,anchor=west,/pgf/number format/fixed},
    enlarge y limits={upper,value=0.5},
    legend cell align=left,
    legend style={
      cells={anchor=east},
      legend pos=outer north east
    }
  ]

  \addplot [style={rred,fill=rred,mark=none},postaction={pattern=north east lines,pattern color=white}] table [col sep=semicolon,y=Ground Truth 0] {Diagrams/Etzioni/testDropSingletonBC.csv};
  \addplot [style={bblue,fill=bblue,mark=none},postaction={pattern=north west lines,pattern color=white}] table [col sep=semicolon,y=Ground Truth Rep 0] {Diagrams/Etzioni/testDropSingletonBC.csv};
  \addplot [style={ggreen,fill=ggreen,mark=none},postaction={pattern=horizontal lines,pattern color=white}] table [col sep=semicolon,y=fMeasure 0] {Diagrams/Etzioni/testDropSingletonBC.csv};

  \legend{Ground truth 0,Ground truth rep 0, F-Measure 0}
  
  \end{axis}
\end{tikzpicture}
  \end{center}
  \caption{Performance of the \CTC algorithm for exclusion and inclusion of singleton base clusters.}
  \label{diag:dropsingletonbcetzioni}
\end{diagram}

% Drop one word clusters test
\begin{diagram}[H]
  \begin{center}
\begin{tikzpicture}
  \begin{axis}[
    width  = 0.8*\textwidth,
    height = 4.55cm,
    % major x tick style = transparent,
    ybar=2*\pgflinewidth,
    bar width=8pt,
    ymajorgrids = true,
    ylabel = {Score},
    xlabel = {Drop one word clusters?},
    symbolic x coords={0,1},
    xtick = data,
    scaled y ticks = false,
    enlarge x limits=0.25,
    ymin=0,
    nodes near coords,
    nodes near coords align={horizontal},
    every node near coord/.append style={font=\tiny,rotate=90,color=black,anchor=west,/pgf/number format/fixed},
    enlarge y limits={upper,value=0.5},
    legend cell align=left,
    legend style={
      cells={anchor=east},
      legend pos=outer north east
    }
  ]

  \addplot [style={rred,fill=rred,mark=none},postaction={pattern=north east lines,pattern color=white}] table [col sep=semicolon,y=Ground Truth 0] {Diagrams/Etzioni/testDropOneWordClusters.csv};
  \addplot [style={bblue,fill=bblue,mark=none},postaction={pattern=north west lines,pattern color=white}] table [col sep=semicolon,y=Ground Truth Rep 0] {Diagrams/Etzioni/testDropOneWordClusters.csv};
  \addplot [style={ggreen,fill=ggreen,mark=none},postaction={pattern=horizontal lines,pattern color=white}] table [col sep=semicolon,y=fMeasure 0] {Diagrams/Etzioni/testDropOneWordClusters.csv};

  \legend{Ground truth 0,Ground truth rep 0, F-Measure 0}
  
  \end{axis}
\end{tikzpicture}
  \end{center}
  \caption{Performance of the \CTC algorithm on exclusion and inclusion of one word clusters.}
  \label{diag:droponewordclustersetzioni}
\end{diagram}

% Sort descending test
\begin{diagram}[H]
  \begin{center}
\begin{tikzpicture}
  \begin{axis}[
    width  = 0.8*\textwidth,
    height = 4.55cm,
    % major x tick style = transparent,
    ybar=2*\pgflinewidth,
    bar width=8pt,
    ymajorgrids = true,
    ylabel = {Score},
    xlabel = {Sort descending?},
    symbolic x coords={0,1},
    xtick = data,
    scaled y ticks = false,
    enlarge x limits=0.25,
    ymin=0,
    nodes near coords,
    nodes near coords align={horizontal},
    every node near coord/.append style={font=\tiny,rotate=90,color=black,anchor=west,/pgf/number format/fixed},
    enlarge y limits={upper,value=0.5},
    legend cell align=left,
    legend style={
      cells={anchor=east},
      legend pos=outer north east
    }
  ]

  \addplot [style={rred,fill=rred,mark=none},postaction={pattern=north east lines,pattern color=white}] table [col sep=semicolon,y=Ground Truth 0] {Diagrams/Etzioni/testSortDescending.csv};
  \addplot [style={bblue,fill=bblue,mark=none},postaction={pattern=north west lines,pattern color=white}] table [col sep=semicolon,y=Ground Truth Rep 0] {Diagrams/Etzioni/testSortDescending.csv};
  \addplot [style={ggreen,fill=ggreen,mark=none},postaction={pattern=horizontal lines,pattern color=white}] table [col sep=semicolon,y=fMeasure 0] {Diagrams/Etzioni/testSortDescending.csv};

  \legend{Ground truth 0,Ground truth rep 0, F-Measure 0}
  
  \end{axis}
\end{tikzpicture}
  \end{center}
  \caption{Performance of the \CTC algorithm when base clusters are sorted in descending and acending order.}
  \label{diag:sortdescendingetzioni}
\end{diagram}

% Text amount
\begin{diagram}[H]
  \begin{center}
\begin{tikzpicture}
  \begin{axis}[
    width  = 0.8*\textwidth,
    height = 4.55cm,
    % major x tick style = transparent,
    xlabel = {Article Text Amount},
    xmin=0,
    xmax=1,
    ylabel = {Score},
    %xtick = data,
    ymin=0,
    legend cell align=left,
    legend style={
      cells={anchor=east},
      legend pos=outer north east
    }
  ]

  \addplot+ [style={rred,mark size=1.5}] table [col sep=semicolon,y=Ground Truth 0,x=Article Text Amount] {Diagrams/Etzioni/testArticleTextAmount.csv};
  \addplot+ [style={bblue,mark size=1.5}] table [col sep=semicolon,y=Ground Truth Rep 0,x=Article Text Amount] {Diagrams/Etzioni/testArticleTextAmount.csv};
  \addplot+ [style={ggreen,mark=triangle*,mark size=1.5}] table [col sep=semicolon,y=fMeasure 0,x=Article Text Amount] {Diagrams/Etzioni/testArticleTextAmount.csv};

  \legend{Ground truth 0,Ground truth rep 0, F-Measure 0}
  
  \end{axis}
\end{tikzpicture}
  \end{center}
  \caption{Performance of the \CTC algorithm for different amounts of article text.}
  \label{diag:textamountetzioni}
\end{diagram}

% TEXT TYPE TESTS
\begin{diagram}[H]
  \begin{center}
\begin{tikzpicture}
  \begin{axis}[
    width  = 0.8*\textwidth,
    height = 4.55cm,
    % major x tick style = transparent,
    ybar=2*\pgflinewidth,
    bar width=6pt,
    ymajorgrids = true,
    ylabel = {Score},
    symbolic x coords={All,Frontpage,Article sans bread text,Article with bread text,Article text},
    x tick label style={font=\small,text width=1.7cm,align=center},
    xtick = data,
    xlabel = {Text types included},
    scaled y ticks = false,
    enlarge x limits=0.10,
    ymin=0,
    nodes near coords,
    nodes near coords align={horizontal},
    every node near coord/.append style={font=\tiny,rotate=90,color=black,anchor=west,/pgf/number format/fixed},
    enlarge y limits={upper,value=0.5},
    legend cell align=left,
    legend style={
      cells={anchor=east},
      legend pos=outer north east
    }
  ]

  \addplot [style={rred,fill=rred,mark=none},postaction={pattern=north east lines,pattern color=white}] table [col sep=semicolon,y=Ground Truth 0] {Diagrams/Etzioni/testTextTypes.csv};
  \addplot [style={bblue,fill=bblue,mark=none},postaction={pattern=north west lines,pattern color=white}] table [col sep=semicolon,y=Ground Truth Rep 0] {Diagrams/Etzioni/testTextTypes.csv};
  \addplot [style={ggreen,fill=ggreen,mark=none},postaction={pattern=horizontal lines,pattern color=white}] table [col sep=semicolon,y=fMeasure 0] {Diagrams/Etzioni/testTextTypes.csv};

  \legend{Ground truth 0,Ground truth rep 0, F-Measure 0}
  
  \end{axis}
\end{tikzpicture}
  \end{center}
  \caption{Performance of the \CTC algorithm for inclusion of different types of texts.}
  \label{diag:texttypesetzioni}
\end{diagram}

% SIMILARITY METHODS TESTS
\begin{diagram}[H]
  \begin{center}
\begin{tikzpicture}
  \begin{axis}[
    width  = 0.8*\textwidth,
    height = 4.55cm,
    % major x tick style = transparent,
    ybar=2*\pgflinewidth,
    bar width=8pt,
    ymajorgrids = true,
    ylabel = {Score},
    xlabel = {Similarity methods},
    symbolic x coords={Etzioni,Jaccard,Cosine,Amendment1C},
    xtick = data,
    scaled y ticks = false,
    enlarge x limits=0.20,
    ymin=0,
    nodes near coords,
    nodes near coords align={horizontal},
    every node near coord/.append style={font=\tiny,rotate=90,color=black,anchor=west, /pgf/number format/fixed},
    enlarge y limits={upper,value=0.5},
    legend cell align=left,
    legend style={
      cells={anchor=east},
      legend pos=outer north east
    }
  ]

  \addplot [style={rred,fill=rred,mark=none},postaction={pattern=north east lines,pattern color=white}] table [col sep=semicolon,y=Ground Truth 0] {Diagrams/Etzioni/testSimilarityMethods.csv};
  \addplot [style={bblue,fill=bblue,mark=none},postaction={pattern=north west lines,pattern color=white}] table [col sep=semicolon,y=Ground Truth Rep 0] {Diagrams/Etzioni/testSimilarityMethods.csv};
  \addplot [style={ggreen,fill=ggreen,mark=none},postaction={pattern=horizontal lines,pattern color=white}] table [col sep=semicolon,y=fMeasure 0] {Diagrams/Etzioni/testSimilarityMethods.csv};

  \legend{Ground truth 0,Ground truth rep 0, F-Measure 0}
  
  \end{axis}
\end{tikzpicture}
  \end{center}
  \caption{Performance of the \CTC algorithm for different similarity methods.}
  \label{diag:similaritymethodsetzioni}
\end{diagram}

% Etzioni THRESHOLD
\begin{diagram}[H]
  \begin{center}
\begin{tikzpicture}
  \begin{axis}[
    % Sizing
    width  = 0.8*\textwidth,
    height = 4.55cm,
    % Data
    xlabel = {Etzioni Similarity Threshold},
    xmin=0,
    xmax=1,
    ymin=0,
    % Labeling
    ylabel = {Score},
    legend cell align=left,
    legend style={
      cells={anchor=east},
      legend pos=outer north east
    }
  ]

  \addplot+ [style={rred,mark size=1.5}] table [col sep=semicolon,y=Ground Truth 0,x=Threshold] {Diagrams/Etzioni/testEtzioniSimilarity.csv};
  \addplot+ [style={bblue,mark size=1.5}] table [col sep=semicolon,y=Ground Truth Rep 0,x=Threshold] {Diagrams/Etzioni/testEtzioniSimilarity.csv};
  \addplot+ [style={ggreen,mark=triangle*,mark size=1.5}] table [col sep=semicolon,y=fMeasure 0,x=Threshold] {Diagrams/Etzioni/testEtzioniSimilarity.csv};

  \legend{Ground truth 0,Ground truth rep 0, F-Measure 0}
  
  \end{axis}
\end{tikzpicture}
  \end{center}
  \caption{Performance of the \CTC algorithm for different Etzioni similarity thresholds.}
  \label{diag:etzionithresholdetzioni}
\end{diagram}

% JACCARD THRESHOLD
\begin{diagram}[H]
  \begin{center}
\begin{tikzpicture}
  \begin{axis}[
    % Sizing
    width  = 0.8*\textwidth,
    height = 4.55cm,
    % Data
    xlabel = {Jaccard Coefficient Threshold},
    xmin=0,
    xmax=1,
    ymin=0,
    % Labeling
    ylabel = {Score},
    legend cell align=left,
    legend style={
      cells={anchor=east},
      legend pos=outer north east
    }
  ]

  \addplot+ [style={rred,mark size=1.5}] table [col sep=semicolon,y=Ground Truth 0,x=Threshold] {Diagrams/Etzioni/testJaccardSimilarity.csv};
  \addplot+ [style={bblue,mark size=1.5}] table [col sep=semicolon,y=Ground Truth Rep 0,x=Threshold] {Diagrams/Etzioni/testJaccardSimilarity.csv};
  \addplot+ [style={ggreen,mark=triangle*,mark size=1.5}] table [col sep=semicolon,y=fMeasure 0,x=Threshold] {Diagrams/Etzioni/testJaccardSimilarity.csv};

  \legend{Ground truth 0,Ground truth rep 0, F-Measure 0}
  
  \end{axis}
\end{tikzpicture}
  \end{center}
  \caption{Performance of the \CTC algorithm for different Jaccard Coefficient thresholds.}
  \label{diag:jaccardthresholdetzioni}
\end{diagram}

% COSINE THRESHOLD
\begin{diagram}[H]
  \begin{center}
\begin{tikzpicture}
  \begin{axis}[
    % Sizing
    width  = 0.8*\textwidth,
    height = 4.55cm,
    % Data
    xlabel = {Cosine Threshold},
    xmin=0,
    xmax=1,
    ymin=0,
    % Labeling
    ylabel = {Score},
    legend cell align=left,
    legend style={
      cells={anchor=east},
      legend pos=outer north east
    }
  ]

  \addplot+ [style={rred,mark size=1.5}] table [col sep=semicolon,y=Ground Truth 0,x=Cosine Threshold] {Diagrams/Etzioni/testCosineSimilarity.csv};
  \addplot+ [style={bblue,mark size=1.5}] table [col sep=semicolon,y=Ground Truth Rep 0,x=Cosine Threshold] {Diagrams/Etzioni/testCosineSimilarity.csv};
  \addplot+ [style={ggreen,mark=triangle*,mark size=1.5}] table [col sep=semicolon,y=fMeasure 0,x=Cosine Threshold] {Diagrams/Etzioni/testCosineSimilarity.csv};

  \legend{Ground truth 0,Ground truth rep 0, F-Measure 0}
  
  \end{axis}
\end{tikzpicture}
  \end{center}
  \caption{Performance of the \CTC algorithm for different Cosine Similarity thresholds.}
  \label{diag:cosinethresholdetzioni}
\end{diagram}

% Average corpus frequency limit
\begin{diagram}[H]
  \begin{center}
\begin{tikzpicture}
  \begin{axis}[
    width  = 0.8*\textwidth,
    height = 4.55cm,
    % major x tick style = transparent,
    xlabel = {Avg corpus frequency limit},
    ylabel = {Score},
    %xtick = data,
    ymin=0,
    xmin=0,
    xmax=500,
    legend cell align=left,
    legend style={
      cells={anchor=east},
      legend pos=outer north east
    }
  ]

  \addplot+ [style={rred,mark size=1.5}] table [col sep=semicolon,y=Ground Truth 0,x=Avg CF limit] {Diagrams/Etzioni/testAmendment1CSimilarityAvgCF.csv};
  \addplot+ [style={bblue,mark size=1.5}] table [col sep=semicolon,y=Ground Truth Rep 0,x=Avg CF limit] {Diagrams/Etzioni/testAmendment1CSimilarityAvgCF.csv};
  \addplot+ [style={ggreen,mark=triangle*,mark size=1.5}] table [col sep=semicolon,y=fMeasure 0,x=Avg CF limit] {Diagrams/Etzioni/testAmendment1CSimilarityAvgCF.csv};

  \legend{Ground truth 0,Ground truth rep 0, F-Measure 0}
  
  \end{axis}
\end{tikzpicture}
  \end{center}
  \caption{Performance of the \CTC algorithm for different limits on max average corpus frequency in Amendment1C.}
  \label{diag:avgcfamendment1etzioni}
\end{diagram}

% Base cluster intersect min limit
\begin{diagram}[H]
  \begin{center}
\begin{tikzpicture}
  \begin{axis}[
    width  = 0.8*\textwidth,
    height = 4.55cm,
    % major x tick style = transparent,
    xlabel = {Min label intersect limit},
    ylabel = {Score},
    %xtick = data,
    ymin=0,
    xmin=0,
    xmax=50,
    legend cell align=left,
    legend style={
      cells={anchor=east},
      legend pos=outer north east
    }
  ]

  \addplot+ [style={rred,mark size=1.5}] table [col sep=semicolon,y=Ground Truth 0,x=Min intersect limit] {Diagrams/Etzioni/testAmendment1CSimilarityIntersect.csv};
  \addplot+ [style={bblue,mark size=1.5}] table [col sep=semicolon,y=Ground Truth Rep 0,x=Min intersect limit] {Diagrams/Etzioni/testAmendment1CSimilarityIntersect.csv};
  \addplot+ [style={ggreen,mark=triangle*,mark size=1.5}] table [col sep=semicolon,y=fMeasure 0,x=Min intersect limit] {Diagrams/Etzioni/testAmendment1CSimilarityIntersect.csv};

  \legend{Ground truth 0,Ground truth rep 0, F-Measure 0}
  
  \end{axis}
\end{tikzpicture}
  \end{center}
  \caption{Performance of the \CTC algorithm for different minimum limits on base cluster label intersect in Amendment1C.}
  \label{diag:minintersectamendment1cetzioni}
\end{diagram}

\section{\supervisor parameters}

Tje following results were obtained by applying the LII groups parameters as base parameters for the incremental tests.

%!TEX root = ../Thesis.tex
% Incremental tests using Richard's parameters as base


\setstretch{1}
% Define bar chart colors
\definecolor{bblue}{HTML}{3366CC}
\definecolor{rred}{HTML}{DC3912}
\definecolor{oorange}{HTML}{FF9900}
\definecolor{ggreen}{HTML}{109618}
\definecolor{ppurple}{HTML}{990099}
\definecolor{tteal}{HTML}{0099C6}

% TREE TYPE TESTS
\begin{figure}[H]
  \begin{center}
\begin{tikzpicture}
  \begin{axis}[
    width  = 0.8*\textwidth,
    height = 4.55cm,
    % major x tick style = transparent,
    ybar=2*\pgflinewidth,
    bar width=6pt,
    ymajorgrids = true,
    ylabel = {Score},
    symbolic x coords={Suffix,Midslice,Rangeslice 0.1-1.0,5-slice},
    xtick = data,
    scaled y ticks = false,
    enlarge x limits=0.15,
    ymin=0,
    nodes near coords,
    nodes near coords align={horizontal},
    every node near coord/.append style={font=\tiny,rotate=90,color=black,anchor=west,/pgf/number format/fixed},
    enlarge y limits={upper,value=0.5},
    legend cell align=left,
    legend style={
      cells={anchor=east},
      legend pos=outer north east
    }
  ]

  \addplot [style={bblue,fill=bblue,mark=none}] table [col sep=semicolon,y=Precision] {Diagrams/Richard/testTreesRichard.csv};
  \addplot [style={rred,fill=rred,mark=none}] table [col sep=semicolon,y=Recall] {Diagrams/Richard/testTreesRichard.csv};
  \addplot [style={oorange,fill=oorange,mark=none}] table [col sep=semicolon,y=F-Measure] {Diagrams/Richard/testTreesRichard.csv};
  \addplot [style={ggreen,fill=ggreen,mark=none}] table [col sep=semicolon,y=Ground Truth 0] {Diagrams/Richard/testTreesRichard.csv};
  \addplot [style={ppurple,fill=ppurple,mark=none}] table [col sep=semicolon,y=Ground Truth Rep 0] {Diagrams/Richard/testTreesRichard.csv};
  \addplot [style={tteal,fill=tteal,mark=none}] table [col sep=semicolon,y=fMeasure 0] {Diagrams/Richard/testTreesRichard.csv};

  \legend{Precision,Recall,F-Measure,Ground truth 0,Ground truth rep 0, F-Measure 0}
  
  \end{axis}
\end{tikzpicture}
  \end{center}
  \caption{Performance of the \CTC algorithm for different expansion techniques.}
  \label{diag:treetypesrichard}
\end{figure}

% N-SLICE
\begin{figure}[H]
  \begin{center}
\begin{tikzpicture}
  \begin{axis}[
    width  = 0.8*\textwidth,
    height = 4.55cm,
    % major x tick style = transparent,
    xlabel = {N-Slice length},
    ylabel = {Score},
    %xtick = data,
    ymin=0,
    legend cell align=left,
    legend style={
      cells={anchor=east},
      legend pos=outer north east
    }
  ]

  \addplot [style={bblue,mark=none}] table [col sep=semicolon,y=Precision,x=n-slice] {Diagrams/Richard/testNSlices.csv};
  \addplot [style={rred,mark=none}] table [col sep=semicolon,y=Recall,x=n-slice] {Diagrams/Richard/testNSlices.csv};
  \addplot [style={oorange,mark=none}] table [col sep=semicolon,y=F-Measure,x=n-slice] {Diagrams/Richard/testNSlices.csv};
  \addplot [style={ggreen,mark=none}] table [col sep=semicolon,y=Ground Truth 0,x=n-slice] {Diagrams/Richard/testNSlices.csv};
  \addplot [style={ppurple,mark=none}] table [col sep=semicolon,y=Ground Truth Rep 0,x=n-slice] {Diagrams/Richard/testNSlices.csv};
  \addplot [style={tteal,mark=none}] table [col sep=semicolon,y=fMeasure 0,x=n-slice] {Diagrams/Richard/testNSlices.csv};

  \legend{Precision,Recall,F-Measure,Ground truth 0,Ground truth rep 0, F-Measure 0}
  
  \end{axis}
\end{tikzpicture}
  \end{center}
  \caption{Performance results of the \CTC algorithm for different lengths of n-slice expansion.}
  \label{diag:nslicesrichard}
\end{figure}

% RANGE SLICE TEST
\begin{figure}[H]
  \begin{center}
\begin{tikzpicture}
  \begin{axis}[
    width  = 0.8*\textwidth,
    height = 4.55cm,
    % major x tick style = transparent,
    ybar=2*\pgflinewidth,
    bar width=5pt,
    ymajorgrids = true,
    ylabel = {Scores},
    xlabel = {Range slice length},
    symbolic x coords={0.0-1.0,0.1-0.9,0.2-0.8,0.3-0.7,0.4-0.6,0.5-0.5},
    xtick = data,
    scaled y ticks = false,
    enlarge x limits=0.10,
    ymin=0,
    ymax=0.6,
    nodes near coords,
    nodes near coords align={horizontal},
    every node near coord/.append style={font=\tiny,rotate=90,color=black,anchor=west,/pgf/number format/fixed},
    enlarge y limits={upper,value=0.5},
    legend cell align=left,
    legend style={
      cells={anchor=east},
      legend pos=outer north east
    }
  ]

  \addplot [style={bblue,fill=bblue,mark=none}] table [col sep=semicolon,y=Precision] {Diagrams/Richard/testRangeSlices.csv};
  \addplot [style={rred,fill=rred,mark=none}] table [col sep=semicolon,y=Recall] {Diagrams/Richard/testRangeSlices.csv};
  \addplot [style={oorange,fill=oorange,mark=none}] table [col sep=semicolon,y=F-Measure] {Diagrams/Richard/testRangeSlices.csv};
  \addplot [style={ggreen,fill=ggreen,mark=none}] table [col sep=semicolon,y=Ground Truth 0] {Diagrams/Richard/testRangeSlices.csv};
  \addplot [style={ppurple,fill=ppurple,mark=none}] table [col sep=semicolon,y=Ground Truth Rep 0] {Diagrams/Richard/testRangeSlices.csv};
  \addplot [style={tteal,fill=tteal,mark=none}] table [col sep=semicolon,y=fMeasure 0] {Diagrams/Richard/testRangeSlices.csv};

  \legend{Precision,Recall,F-Measure,Ground truth 0,Ground truth rep 0, F-Measure 0}
  
  \end{axis}
\end{tikzpicture}
  \end{center}
  \caption{Performance of the \CTC algorithm for different ranges of range-slice expansion.}
  \label{diag:rangelicesrichard}
\end{figure}

% NUMBER OF TOP BASE CLUSTERS
\begin{figure}[H]
  \begin{center}
\begin{tikzpicture}
  \begin{axis}[
    width  = 0.8*\textwidth,
    height = 4.55cm,
    % major x tick style = transparent,
    xlabel = {Base cluster amount},
    ylabel = {Score},
    %xtick = data,
    ymin=0,
    xmin=0,
    legend cell align=left,
    legend style={
      cells={anchor=east},
      legend pos=outer north east
    }
  ]

  \addplot [style={bblue,mark=none}] table [col sep=semicolon,y=Precision,x=Basecluster-amount] {Diagrams/Richard/testBaseClusterAmounts.csv};
  \addplot [style={rred,mark=none}] table [col sep=semicolon,y=Recall,x=Basecluster-amount] {Diagrams/Richard/testBaseClusterAmounts.csv};
  \addplot [style={oorange,mark=none}] table [col sep=semicolon,y=F-Measure,x=Basecluster-amount] {Diagrams/Richard/testBaseClusterAmounts.csv};
  \addplot [style={ggreen,mark=none}] table [col sep=semicolon,y=Ground Truth 0,x=Basecluster-amount] {Diagrams/Richard/testBaseClusterAmounts.csv};
  \addplot [style={ppurple,mark=none}] table [col sep=semicolon,y=Ground Truth Rep 0,x=Basecluster-amount] {Diagrams/Richard/testBaseClusterAmounts.csv};
  \addplot [style={tteal,mark=none}] table [col sep=semicolon,y=fMeasure 0,x=Basecluster-amount] {Diagrams/Richard/testBaseClusterAmounts.csv};

  \legend{Precision,Recall,F-Measure,Ground truth 0,Ground truth rep 0, F-Measure 0}
  
  \end{axis}
\end{tikzpicture}
  \end{center}
  \caption{Performance of the \CTC algorithm fordifferent limits on top base clusters amount.}
  \label{diag:topbaseclustersrichard}
\end{figure}

% MIN TERM OCCURRENCE
\begin{figure}[H]
  \begin{center}
\begin{tikzpicture}
  \begin{axis}[
    width  = 0.8*\textwidth,
    height = 4.55cm,
    % major x tick style = transparent,
    xlabel = {Min term occurrence},
    ylabel = {Score},
    %xtick = data,
    ymin=0,
    xmin=0,
    xmax=200,
    legend cell align=left,
    legend style={
      cells={anchor=east},
      legend pos=outer north east
    }
  ]

  \addplot [style={bblue,mark=none}] table [col sep=semicolon,y=Precision,x=Min Term Occurrence] {Diagrams/Richard/testMinTermOccurrence.csv};
  \addplot [style={rred,mark=none}] table [col sep=semicolon,y=Recall,x=Min Term Occurrence] {Diagrams/Richard/testMinTermOccurrence.csv};
  \addplot [style={oorange,mark=none}] table [col sep=semicolon,y=F-Measure,x=Min Term Occurrence] {Diagrams/Richard/testMinTermOccurrence.csv};
  \addplot [style={ggreen,mark=none}] table [col sep=semicolon,y=Ground Truth 0,x=Min Term Occurrence] {Diagrams/Richard/testMinTermOccurrence.csv};
  \addplot [style={ppurple,mark=none}] table [col sep=semicolon,y=Ground Truth Rep 0,x=Min Term Occurrence] {Diagrams/Richard/testMinTermOccurrence.csv};
  \addplot [style={tteal,mark=none}] table [col sep=semicolon,y=fMeasure 0,x=Min Term Occurrence] {Diagrams/Richard/testMinTermOccurrence.csv};

  \legend{Precision,Recall,F-Measure,Ground truth 0,Ground truth rep 0, F-Measure 0}
  
  \end{axis}
\end{tikzpicture}
  \end{center}
  \caption{Performance of the \CTC algorithm for different limits on minimal term occurrence in collection.}
  \label{diag:mintermoccurrencesrichard}
\end{figure}

% MAX TERM RATIO
\begin{figure}[H]
  \begin{center}
\begin{tikzpicture}
  \begin{axis}[
    width  = 0.8*\textwidth,
    height = 4.55cm,
    % major x tick style = transparent,
    xlabel = {Max term ratio},
    ylabel = {Score},
    %xtick = data,
    ymin=0,
    xmin=0.1,
    xmax=1,
    legend cell align=left,
    legend style={
      cells={anchor=east},
      legend pos=outer north east
    }
  ]

  \addplot [style={bblue,mark=none}] table [col sep=semicolon,y=Precision,x=Max Term Ratio] {Diagrams/Richard/testMaxTermRatio.csv};
  \addplot [style={rred,mark=none}] table [col sep=semicolon,y=Recall,x=Max Term Ratio] {Diagrams/Richard/testMaxTermRatio.csv};
  \addplot [style={oorange,mark=none}] table [col sep=semicolon,y=F-Measure,x=Max Term Ratio] {Diagrams/Richard/testMaxTermRatio.csv};
  \addplot [style={ggreen,mark=none}] table [col sep=semicolon,y=Ground Truth 0,x=Max Term Ratio] {Diagrams/Richard/testMaxTermRatio.csv};
  \addplot [style={ppurple,mark=none}] table [col sep=semicolon,y=Ground Truth Rep 0,x=Max Term Ratio] {Diagrams/Richard/testMaxTermRatio.csv};
  \addplot [style={tteal,mark=none}] table [col sep=semicolon,y=fMeasure 0,x=Max Term Ratio] {Diagrams/Richard/testMaxTermRatio.csv};

  \legend{Precision,Recall,F-Measure,Ground truth 0,Ground truth rep 0, F-Measure 0}
  
  \end{axis}
\end{tikzpicture}
  \end{center}
  \caption{Performance of the \CTC algorithm for different limits on max term ratio in collection.}
  \label{diag:maxtermratiorichard}
\end{figure}

% Min Limit BC Score
\begin{figure}[H]
  \begin{center}
\begin{tikzpicture}
  \begin{axis}[
    width  = 0.8*\textwidth,
    height = 4.55cm,
    % major x tick style = transparent,
    xlabel = {Min Limit BC Score},
    ylabel = {Score},
    %xtick = data,
    ymin=0,
    xmin=0,
    xmax=15,
    legend cell align=left,
    legend style={
      cells={anchor=east},
      legend pos=outer north east
    }
  ]

  \addplot [style={bblue,mark=none}] table [col sep=semicolon,y=Precision,x=Min Limit] {Diagrams/Richard/testMinLimitBC.csv};
  \addplot [style={rred,mark=none}] table [col sep=semicolon,y=Recall,x=Min Limit] {Diagrams/Richard/testMinLimitBC.csv};
  \addplot [style={oorange,mark=none}] table [col sep=semicolon,y=F-Measure,x=Min Limit] {Diagrams/Richard/testMinLimitBC.csv};
  \addplot [style={ggreen,mark=none}] table [col sep=semicolon,y=Ground Truth 0,x=Min Limit] {Diagrams/Richard/testMinLimitBC.csv};
  \addplot [style={ppurple,mark=none}] table [col sep=semicolon,y=Ground Truth Rep 0,x=Min Limit] {Diagrams/Richard/testMinLimitBC.csv};
  \addplot [style={tteal,mark=none}] table [col sep=semicolon,y=fMeasure 0,x=Min Limit] {Diagrams/Richard/testMinLimitBC.csv};

  \legend{Precision,Recall,F-Measure,Ground truth 0,Ground truth rep 0, F-Measure 0}
  
  \end{axis}
\end{tikzpicture}
  \end{center}
  \caption{Performance of the \CTC algorithm for different min limit values for base cluster score with unbounded max limit (max limit = length of longest label).}
  \label{diag:minlimitbcscorerichard}
\end{figure}

% Max Limit BC Score
\begin{figure}[H]
  \begin{center}
\begin{tikzpicture}
  \begin{axis}[
    width  = 0.8*\textwidth,
    height = 4.55cm,
    % major x tick style = transparent,
    xlabel = {Max Limit BC Score},
    ylabel = {Score},
    %xtick = data,
    ymin=0,
    xmin=3,
    xmax=15,
    legend cell align=left,
    legend style={
      cells={anchor=east},
      legend pos=outer north east
    }
  ]

  \addplot [style={bblue,mark=none}] table [col sep=semicolon,y=Precision,x=Max Limit] {Diagrams/Richard/testMaxLimitBC.csv};
  \addplot [style={rred,mark=none}] table [col sep=semicolon,y=Recall,x=Max Limit] {Diagrams/Richard/testMaxLimitBC.csv};
  \addplot [style={oorange,mark=none}] table [col sep=semicolon,y=F-Measure,x=Max Limit] {Diagrams/Richard/testMaxLimitBC.csv};
  \addplot [style={ggreen,mark=none}] table [col sep=semicolon,y=Ground Truth 0,x=Max Limit] {Diagrams/Richard/testMaxLimitBC.csv};
  \addplot [style={ppurple,mark=none}] table [col sep=semicolon,y=Ground Truth Rep 0,x=Max Limit] {Diagrams/Richard/testMaxLimitBC.csv};
  \addplot [style={tteal,mark=none}] table [col sep=semicolon,y=fMeasure 0,x=Max Limit] {Diagrams/Richard/testMaxLimitBC.csv};

  \legend{Precision,Recall,F-Measure,Ground truth 0,Ground truth rep 0, F-Measure 0}
  
  \end{axis}
\end{tikzpicture}
  \end{center}
  \caption{Performance of the \CTC algorithm for different max limit values for base cluster score. Min limit set to \protect\citeauthor{Oren1998} default.}
  \label{diag:maxlimitbcscorerichard}
\end{figure}

% Drop singleton bc test
\begin{figure}[H]
  \begin{center}
\begin{tikzpicture}
  \begin{axis}[
    width  = 0.8*\textwidth,
    height = 4.55cm,
    % major x tick style = transparent,
    ybar=2*\pgflinewidth,
    bar width=8pt,
    ymajorgrids = true,
    ylabel = {Score},
    symbolic x coords={0,1},
    xtick = data,
    scaled y ticks = false,
    enlarge x limits=0.25,
    ymin=0,
    nodes near coords,
    nodes near coords align={horizontal},
    every node near coord/.append style={font=\tiny,rotate=90,color=black,anchor=west,/pgf/number format/fixed},
    enlarge y limits={upper,value=0.5},
    legend cell align=left,
    legend style={
      cells={anchor=east},
      legend pos=outer north east
    }
  ]

  \addplot [style={bblue,fill=bblue,mark=none}] table [col sep=semicolon,y=Precision] {Diagrams/Richard/testDropSingletonBC.csv};
  \addplot [style={rred,fill=rred,mark=none}] table [col sep=semicolon,y=Recall] {Diagrams/Richard/testDropSingletonBC.csv};
  \addplot [style={oorange,fill=oorange,mark=none}] table [col sep=semicolon,y=F-Measure] {Diagrams/Richard/testDropSingletonBC.csv};
  \addplot [style={ggreen,fill=ggreen,mark=none}] table [col sep=semicolon,y=Ground Truth 0] {Diagrams/Richard/testDropSingletonBC.csv};
  \addplot [style={ppurple,fill=ppurple,mark=none}] table [col sep=semicolon,y=Ground Truth Rep 0] {Diagrams/Richard/testDropSingletonBC.csv};
  \addplot [style={tteal,fill=tteal,mark=none}] table [col sep=semicolon,y=fMeasure 0] {Diagrams/Richard/testDropSingletonBC.csv};

  \legend{Precision,Recall,F-Measure,Ground truth 0,Ground truth rep 0, F-Measure 0}
  
  \end{axis}
\end{tikzpicture}
  \end{center}
  \caption{Performance of the \CTC algorithm for exclusion and inclusion of singleton base clusters.}
  \label{diag:dropsingletonbcrichard}
\end{figure}

% Drop one word clusters test
\begin{figure}[H]
  \begin{center}
\begin{tikzpicture}
  \begin{axis}[
    width  = 0.8*\textwidth,
    height = 4.55cm,
    % major x tick style = transparent,
    ybar=2*\pgflinewidth,
    bar width=8pt,
    ymajorgrids = true,
    ylabel = {Score},
    symbolic x coords={0,1},
    xtick = data,
    scaled y ticks = false,
    enlarge x limits=0.25,
    ymin=0,
    nodes near coords,
    nodes near coords align={horizontal},
    every node near coord/.append style={font=\tiny,rotate=90,color=black,anchor=west,/pgf/number format/fixed},
    enlarge y limits={upper,value=0.5},
    legend cell align=left,
    legend style={
      cells={anchor=east},
      legend pos=outer north east
    }
  ]

  \addplot [style={bblue,fill=bblue,mark=none}] table [col sep=semicolon,y=Precision] {Diagrams/Richard/testDropOneWordClusters.csv};
  \addplot [style={rred,fill=rred,mark=none}] table [col sep=semicolon,y=Recall] {Diagrams/Richard/testDropOneWordClusters.csv};
  \addplot [style={oorange,fill=oorange,mark=none}] table [col sep=semicolon,y=F-Measure] {Diagrams/Richard/testDropOneWordClusters.csv};
  \addplot [style={ggreen,fill=ggreen,mark=none}] table [col sep=semicolon,y=Ground Truth 0] {Diagrams/Richard/testDropOneWordClusters.csv};
  \addplot [style={ppurple,fill=ppurple,mark=none}] table [col sep=semicolon,y=Ground Truth Rep 0] {Diagrams/Richard/testDropOneWordClusters.csv};
  \addplot [style={tteal,fill=tteal,mark=none}] table [col sep=semicolon,y=fMeasure 0] {Diagrams/Richard/testDropOneWordClusters.csv};

  \legend{Precision,Recall,F-Measure,Ground truth 0,Ground truth rep 0, F-Measure 0}
  
  \end{axis}
\end{tikzpicture}
  \end{center}
  \caption{Performance of the \CTC algorithm on exclusion and inclusion of one word clusters.}
  \label{diag:droponewordclustersrichard}
\end{figure}

% Sort descending test
\begin{figure}[H]
  \begin{center}
\begin{tikzpicture}
  \begin{axis}[
    width  = 0.8*\textwidth,
    height = 4.55cm,
    % major x tick style = transparent,
    ybar=2*\pgflinewidth,
    bar width=8pt,
    ymajorgrids = true,
    ylabel = {Score},
    symbolic x coords={0,1},
    xtick = data,
    scaled y ticks = false,
    enlarge x limits=0.25,
    ymin=0,
    nodes near coords,
    nodes near coords align={horizontal},
    every node near coord/.append style={font=\tiny,rotate=90,color=black,anchor=west,/pgf/number format/fixed},
    enlarge y limits={upper,value=0.5},
    legend cell align=left,
    legend style={
      cells={anchor=east},
      legend pos=outer north east
    }
  ]

  \addplot [style={bblue,fill=bblue,mark=none}] table [col sep=semicolon,y=Precision] {Diagrams/Richard/testSortDescending.csv};
  \addplot [style={rred,fill=rred,mark=none}] table [col sep=semicolon,y=Recall] {Diagrams/Richard/testSortDescending.csv};
  \addplot [style={oorange,fill=oorange,mark=none}] table [col sep=semicolon,y=F-Measure] {Diagrams/Richard/testSortDescending.csv};
  \addplot [style={ggreen,fill=ggreen,mark=none}] table [col sep=semicolon,y=Ground Truth 0] {Diagrams/Richard/testSortDescending.csv};
  \addplot [style={ppurple,fill=ppurple,mark=none}] table [col sep=semicolon,y=Ground Truth Rep 0] {Diagrams/Richard/testSortDescending.csv};
  \addplot [style={tteal,fill=tteal,mark=none}] table [col sep=semicolon,y=fMeasure 0] {Diagrams/Richard/testSortDescending.csv};

  \legend{Precision,Recall,F-Measure,Ground truth 0,Ground truth rep 0, F-Measure 0}
  
  \end{axis}
\end{tikzpicture}
  \end{center}
  \caption{Performance of the \CTC algorithm when base clusters are sorted in descending and acending order.}
  \label{diag:sortdescendingrichard}
\end{figure}

% Text amount
\begin{figure}[H]
  \begin{center}
\begin{tikzpicture}
  \begin{axis}[
    width  = 0.8*\textwidth,
    height = 4.55cm,
    % major x tick style = transparent,
    xlabel = {Article Text Amount},
    xmin=0,
    xmax=1,
    ylabel = {Score},
    %xtick = data,
    ymin=0,
    legend cell align=left,
    legend style={
      cells={anchor=east},
      legend pos=outer north east
    }
  ]

  \addplot [style={bblue,mark=none}] table [col sep=semicolon,y=Precision,x=Article Text Amount] {Diagrams/Richard/testArticleTextAmount.csv};
  \addplot [style={rred,mark=none}] table [col sep=semicolon,y=Recall,x=Article Text Amount] {Diagrams/Richard/testArticleTextAmount.csv};
  \addplot [style={oorange,mark=none}] table [col sep=semicolon,y=F-Measure,x=Article Text Amount] {Diagrams/Richard/testArticleTextAmount.csv};
  \addplot [style={ggreen,mark=none}] table [col sep=semicolon,y=Ground Truth 0,x=Article Text Amount] {Diagrams/Richard/testArticleTextAmount.csv};
  \addplot [style={ppurple,mark=none}] table [col sep=semicolon,y=Ground Truth Rep 0,x=Article Text Amount] {Diagrams/Richard/testArticleTextAmount.csv};
  \addplot [style={tteal,mark=none}] table [col sep=semicolon,y=fMeasure 0,x=Article Text Amount] {Diagrams/Richard/testArticleTextAmount.csv};

  \legend{Precision,Recall,F-Measure,Ground truth 0,Ground truth rep 0, F-Measure 0}
  
  \end{axis}
\end{tikzpicture}
  \end{center}
  \caption{Performance of the \CTC algorithm for different amounts of article text.}
  \label{diag:textamountrichard}
\end{figure}

% TEXT TYPE TESTS
\begin{figure}[H]
  \begin{center}
\begin{tikzpicture}
  \begin{axis}[
    width  = 0.8*\textwidth,
    height = 4.55cm,
    % major x tick style = transparent,
    ybar=2*\pgflinewidth,
    bar width=6pt,
    ymajorgrids = true,
    ylabel = {Score},
    xlabel = {Text types included},
    symbolic x coords={All,Frontpage,Article sans bread text,Article with bread text,Article text},
    x tick label style={font=\small,text width=1.7cm,align=center},
    xtick = data,
    scaled y ticks = false,
    enlarge x limits=0.10,
    ymin=0,
    nodes near coords,
    nodes near coords align={horizontal},
    every node near coord/.append style={font=\tiny,rotate=90,color=black,anchor=west,/pgf/number format/fixed},
    enlarge y limits={upper,value=0.5},
    legend cell align=left,
    legend style={
      cells={anchor=east},
      legend pos=outer north east
    }
  ]

  \addplot [style={bblue,fill=bblue,mark=none}] table [col sep=semicolon,y=Precision] {Diagrams/Richard/testTextTypes.csv};
  \addplot [style={rred,fill=rred,mark=none}] table [col sep=semicolon,y=Recall] {Diagrams/Richard/testTextTypes.csv};
  \addplot [style={oorange,fill=oorange,mark=none}] table [col sep=semicolon,y=F-Measure] {Diagrams/Richard/testTextTypes.csv};
  \addplot [style={ggreen,fill=ggreen,mark=none}] table [col sep=semicolon,y=Ground Truth 0] {Diagrams/Richard/testTextTypes.csv};
  \addplot [style={ppurple,fill=ppurple,mark=none}] table [col sep=semicolon,y=Ground Truth Rep 0] {Diagrams/Richard/testTextTypes.csv};
  \addplot [style={tteal,fill=tteal,mark=none}] table [col sep=semicolon,y=fMeasure 0] {Diagrams/Richard/testTextTypes.csv};

  \legend{Precision,Recall,F-Measure,Ground truth 0,Ground truth rep 0, F-Measure 0}
  
  \end{axis}
\end{tikzpicture}
  \end{center}
  \caption{Performance of the \CTC algorithm for inclusion of different types of texts.}
  \label{diag:texttypesrichard}
\end{figure}

% SIMILARITY METHODS TESTS
\begin{figure}[H]
  \begin{center}
\begin{tikzpicture}
  \begin{axis}[
    width  = 0.8*\textwidth,
    height = 4.55cm,
    % major x tick style = transparent,
    ybar=2*\pgflinewidth,
    bar width=8pt,
    ymajorgrids = true,
    ylabel = {Score},
    xlabel = {Similarity methods},
    symbolic x coords={Jaccard,Cosine,Amendment1C},
    xtick = data,
    scaled y ticks = false,
    enlarge x limits=0.20,
    ymin=0,
    nodes near coords,
    nodes near coords align={horizontal},
    every node near coord/.append style={font=\tiny,rotate=90,color=black,anchor=west, /pgf/number format/fixed},
    enlarge y limits={upper,value=0.5},
    legend cell align=left,
    legend style={
      cells={anchor=east},
      legend pos=outer north east
    }
  ]

  \addplot [style={bblue,fill=bblue,mark=none}] table [col sep=semicolon,y=Precision] {Diagrams/Richard/testSimilarityMethods.csv};
  \addplot [style={rred,fill=rred,mark=none}] table [col sep=semicolon,y=Recall] {Diagrams/Richard/testSimilarityMethods.csv};
  \addplot [style={oorange,fill=oorange,mark=none}] table [col sep=semicolon,y=F-Measure] {Diagrams/Richard/testSimilarityMethods.csv};
  \addplot [style={ggreen,fill=ggreen,mark=none}] table [col sep=semicolon,y=Ground Truth 0] {Diagrams/Richard/testSimilarityMethods.csv};
  \addplot [style={ppurple,fill=ppurple,mark=none}] table [col sep=semicolon,y=Ground Truth Rep 0] {Diagrams/Richard/testSimilarityMethods.csv};
  \addplot [style={tteal,fill=tteal,mark=none}] table [col sep=semicolon,y=fMeasure 0] {Diagrams/Richard/testSimilarityMethods.csv};

  \legend{Precision,Recall,F-Measure,Ground truth 0,Ground truth rep 0, F-Measure 0}
  
  \end{axis}
\end{tikzpicture}
  \end{center}
  \caption{Performance of the \CTC algorithm for different similarity methods.}
  \label{diag:similaritymethodsrichard}
\end{figure}

% JACCARD THRESHOLD
\begin{figure}[H]
  \begin{center}
\begin{tikzpicture}
  \begin{axis}[
    % Sizing
    width  = 0.8*\textwidth,
    height = 4.55cm,
    % Data
    xlabel = {Jaccard Coefficient Threshold},
    xmin=0,
    xmax=1,
    ymin=0,
    % Labeling
    ylabel = {Score},
    legend cell align=left,
    legend style={
      cells={anchor=east},
      legend pos=outer north east
    }
  ]

  \addplot [style={bblue,mark=none}] table [col sep=semicolon,y=Precision,x=Threshold] {Diagrams/Richard/testJaccardSimilarity.csv};
  \addplot [style={rred,mark=none}] table [col sep=semicolon,y=Recall,x=Threshold] {Diagrams/Richard/testJaccardSimilarity.csv};
  \addplot [style={oorange,mark=none}] table [col sep=semicolon,y=F-Measure,x=Threshold] {Diagrams/Richard/testJaccardSimilarity.csv};
  \addplot [style={ggreen,mark=none}] table [col sep=semicolon,y=Ground Truth 0,x=Threshold] {Diagrams/Richard/testJaccardSimilarity.csv};
  \addplot [style={ppurple,mark=none}] table [col sep=semicolon,y=Ground Truth Rep 0,x=Threshold] {Diagrams/Richard/testJaccardSimilarity.csv};
  \addplot [style={tteal,mark=none}] table [col sep=semicolon,y=fMeasure 0,x=Threshold] {Diagrams/Richard/testJaccardSimilarity.csv};

  \legend{Precision,Recall,F-Measure,Ground truth 0,Ground truth rep 0, F-Measure 0}
  
  \end{axis}
\end{tikzpicture}
  \end{center}
  \caption{Performance of the \CTC algorithm for different Jaccard Coefficient thresholds.}
  \label{diag:jaccardthresholdrichard}
\end{figure}

% COSINE THRESHOLD
\begin{figure}[H]
  \begin{center}
\begin{tikzpicture}
  \begin{axis}[
    % Sizing
    width  = 0.8*\textwidth,
    height = 4.55cm,
    % Data
    xlabel = {Cosine Threshold},
    xmin=0,
    xmax=1,
    ymin=0,
    % Labeling
    ylabel = {Score},
    legend cell align=left,
    legend style={
      cells={anchor=east},
      legend pos=outer north east
    }
  ]

  \addplot [style={bblue,mark=none}] table [col sep=semicolon,y=Precision,x=Cosine Threshold] {Diagrams/Richard/testCosineSimilarity.csv};
  \addplot [style={rred,mark=none}] table [col sep=semicolon,y=Recall,x=Cosine Threshold] {Diagrams/Richard/testCosineSimilarity.csv};
  \addplot [style={oorange,mark=none}] table [col sep=semicolon,y=F-Measure,x=Cosine Threshold] {Diagrams/Richard/testCosineSimilarity.csv};
  \addplot [style={ggreen,mark=none}] table [col sep=semicolon,y=Ground Truth 0,x=Cosine Threshold] {Diagrams/Richard/testCosineSimilarity.csv};
  \addplot [style={ppurple,mark=none}] table [col sep=semicolon,y=Ground Truth Rep 0,x=Cosine Threshold] {Diagrams/Richard/testCosineSimilarity.csv};
  \addplot [style={tteal,mark=none}] table [col sep=semicolon,y=fMeasure 0,x=Cosine Threshold] {Diagrams/Richard/testCosineSimilarity.csv};

  \legend{Precision,Recall,F-Measure,Ground truth 0,Ground truth rep 0, F-Measure 0}
  
  \end{axis}
\end{tikzpicture}
  \end{center}
  \caption{Performance of the \CTC algorithm for different Cosine Similarity thresholds.}
  \label{diag:cosinethresholdrichard}
\end{figure}

% Average corpus frequency limit
\begin{figure}[H]
  \begin{center}
\begin{tikzpicture}
  \begin{axis}[
    width  = 0.8*\textwidth,
    height = 4.55cm,
    % major x tick style = transparent,
    xlabel = {Avg corpus frequency limit},
    ylabel = {Score},
    %xtick = data,
    ymin=0,
    xmin=0,
    xmax=500,
    legend cell align=left,
    legend style={
      cells={anchor=east},
      legend pos=outer north east
    }
  ]

  \addplot [style={bblue,mark=none}] table [col sep=semicolon,y=Precision,x=Avg CF limit] {Diagrams/Richard/testAmendment1CSimilarityAvgCF.csv};
  \addplot [style={rred,mark=none}] table [col sep=semicolon,y=Recall,x=Avg CF limit] {Diagrams/Richard/testAmendment1CSimilarityAvgCF.csv};
  \addplot [style={oorange,mark=none}] table [col sep=semicolon,y=F-Measure,x=Avg CF limit] {Diagrams/Richard/testAmendment1CSimilarityAvgCF.csv};
  \addplot [style={ggreen,mark=none}] table [col sep=semicolon,y=Ground Truth 0,x=Avg CF limit] {Diagrams/Richard/testAmendment1CSimilarityAvgCF.csv};
  \addplot [style={ppurple,mark=none}] table [col sep=semicolon,y=Ground Truth Rep 0,x=Avg CF limit] {Diagrams/Richard/testAmendment1CSimilarityAvgCF.csv};
  \addplot [style={tteal,mark=none}] table [col sep=semicolon,y=fMeasure 0,x=Avg CF limit] {Diagrams/Richard/testAmendment1CSimilarityAvgCF.csv};

  \legend{Precision,Recall,F-Measure,Ground truth 0,Ground truth rep 0, F-Measure 0}
  
  \end{axis}
\end{tikzpicture}
  \end{center}
  \caption{Performance of the \CTC algorithm for different limits on max average corpus frequency in Amendment1C.}
  \label{diag:avgcfamendment1richard}
\end{figure}

% Base cluster intersect min limit
\begin{figure}[H]
  \begin{center}
\begin{tikzpicture}
  \begin{axis}[
    width  = 0.8*\textwidth,
    height = 4.55cm,
    % major x tick style = transparent,
    xlabel = {Min label intersect limit},
    ylabel = {Score},
    %xtick = data,
    ymin=0,
    xmin=0,
    xmax=50,
    legend cell align=left,
    legend style={
      cells={anchor=east},
      legend pos=outer north east
    }
  ]

  \addplot [style={bblue,mark=none}] table [col sep=semicolon,y=Precision,x=Min intersect limit] {Diagrams/Richard/testAmendment1CSimilarityIntersect.csv};
  \addplot [style={rred,mark=none}] table [col sep=semicolon,y=Recall,x=Min intersect limit] {Diagrams/Richard/testAmendment1CSimilarityIntersect.csv};
  \addplot [style={oorange,mark=none}] table [col sep=semicolon,y=F-Measure,x=Min intersect limit] {Diagrams/Richard/testAmendment1CSimilarityIntersect.csv};
  \addplot [style={ggreen,mark=none}] table [col sep=semicolon,y=Ground Truth 0,x=Min intersect limit] {Diagrams/Richard/testAmendment1CSimilarityIntersect.csv};
  \addplot [style={ppurple,mark=none}] table [col sep=semicolon,y=Ground Truth Rep 0,x=Min intersect limit] {Diagrams/Richard/testAmendment1CSimilarityIntersect.csv};
  \addplot [style={tteal,mark=none}] table [col sep=semicolon,y=fMeasure 0,x=Min intersect limit] {Diagrams/Richard/testAmendment1CSimilarityIntersect.csv};

  \legend{Precision,Recall,F-Measure,Ground truth 0,Ground truth rep 0, F-Measure 0}
  
  \end{axis}
\end{tikzpicture}
  \end{center}
  \caption{Performance of the \CTC algorithm for different minimum limits on base cluster label intersect in Amendment1C.}
  \label{diag:minintersectamendment1crichard}
\end{figure}
%\input{./Appendices/AppendixB}
%\input{./Appendices/AppendixC}

\addtocontents{toc}{\vspace{2em}} % Add a gap in the Contents, for aesthetics

\backmatter

%----------------------------------------------------------------------------------------
%	BIBLIOGRAPHY
%----------------------------------------------------------------------------------------

\label{Bibliography}

%\lhead{\emph{Bibliography}} % Change the page header to say "Bibliography"

\bibliographystyle{apalike} % Use the "unsrtnat" BibTeX style for formatting the Bibliography

\bibliography{bib/library} % The references (bibliography) information are stored in the file named "Bibliography.bib"

\end{document}
